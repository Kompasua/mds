% Math symbols and examples: http://en.wikibooks.org/wiki/LaTeX/Mathematics
% Also useful link: http://en.wikibooks.org/wiki/LaTeX/Advanced_Mathematics

% This document is used for title generating of exercise solutions. 
% Also it used for generating styles and environmentof documents.
% To add this title in document add next line:
% % This document is used for title generating of exercise solutions. 
% Also it used for generating styles and environmentof documents.
% To add this title in document add next line:
% % This document is used for title generating of exercise solutions. 
% Also it used for generating styles and environmentof documents.
% To add this title in document add next line:
% \input{../template.tex}
\documentclass[a4paper]{article}
\usepackage[utf8]{inputenc}
\usepackage[top=1.5cm]{geometry}
\usepackage{amssymb}
\usepackage{enumitem}
\usepackage{amsmath}

\allowdisplaybreaks

\DeclareMathOperator{\ggT}{ggT}
\DeclareMathOperator{\kgV}{kgV}

\title{Mathematik: Diskrete Strukturen \\ \Large Lösungsblatt}
\author{Anton Bubnov, Eugen Kuzmenko}

\documentclass[a4paper]{article}
\usepackage[utf8]{inputenc}
\usepackage[top=1.5cm]{geometry}
\usepackage{amssymb}
\usepackage{enumitem}
\usepackage{amsmath}

\allowdisplaybreaks

\DeclareMathOperator{\ggT}{ggT}
\DeclareMathOperator{\kgV}{kgV}

\title{Mathematik: Diskrete Strukturen \\ \Large Lösungsblatt}
\author{Anton Bubnov, Eugen Kuzmenko}

\documentclass[a4paper]{article}
\usepackage[utf8]{inputenc}
\usepackage[top=1.5cm]{geometry}
\usepackage{amssymb}
\usepackage{enumitem}
\usepackage{amsmath}

\allowdisplaybreaks

\DeclareMathOperator{\ggT}{ggT}
\DeclareMathOperator{\kgV}{kgV}

\title{Mathematik: Diskrete Strukturen \\ \Large Lösungsblatt}
\author{Anton Bubnov, Eugen Kuzmenko}


\begin{document}
    \maketitle
    \section*{Vertiefung:}
    \begin{enumerate}[label=(\alph*)]
        % Task (a)
        \item Welchen Wert hat $\pi^{23}(5)$ f\"ur $\pi=(4 2 5 3 1)(8 6 7)$ ?\\\\
        Da 5 befindet sich in dem Zyklus der Länge 5, gilt $\pi^{23}(5)=\pi^{3}(5)$
        \begin{align*}
	        \pi^{0}(5) &= 5\\
	        \pi^2(5) &= 1\\
	        \pi^3(5) &= 4\\
	        \pi^{23}(5)&=4.
        \end{align*}
        
        % Task (b)
        \item Wie sieht die Permutation (3, 2, 6, 7, 5, 1, 4) in Zyklenschreibweise aus?
        \begin{align*}
            \pi^0(1) &= 1, \quad \pi^1(1) = 3, \quad \pi^2(1)=6, \quad \pi^3(1)=1; \\
            \pi^0(2) &= 2, \quad \pi^1(2) = 2; \\
            \pi^0(3) &= 3, \quad \pi^1(3) = 6, \quad \pi^2(3)=1, \quad \pi^3(3)=3; \\
            \pi^0(4) &= 4, \quad \pi^1(4) = 7, \quad \pi^2(4)=4; \\
            \pi^0(5) &= 5, \quad \pi^1(5) = 5; \\
            \pi^0(6) &= 6, \quad \pi^1(6) = 1, \quad \pi^2(6)=3, \quad \pi^3(6)=6; \\
            \pi^0(7) &= 7, \quad \pi^1(7) = 4, \quad \pi^2(7)=7;
        \end{align*}
       	\[\textrm{Zyklenschreibweise: } (136)(2)(47)(5)\]
        
        % Task (c)
        \item Wie sieht die Permutation (8 1 5 3)(2 4)(6 7) in Tupelschreibweise aus?
        \begin{align*}
        	\pi(8) &=1 \quad \pi(1)= 5 \quad \pi(5)=3 \quad \pi(3)=8\\
        	\pi(2) &=4 \quad \pi(4)=2\\
        	\pi(7) &=7 \quad \pi(7)=6
        \end{align*}
        \[\textrm{Tupelschreibweise: } (5,4,8,2,3,7,6,1)\]
        
        % Task (d)
        \item Welchen Zyklentyp besitzt die Permutation (2, 4, 1, 3, 5, 8, 6, 9, 7)?\\\\
        Es gibt folgende Zyklen: (2 4 3 1)(5)(8 9 7 6), der Zyklentyp ist (4, 4, 1), oder $4^2 1^1$.
        % Task (e)
        \item Wie viele Permutationen von $n$ Elementen mit dem Zyklentyp 
        $(3, 1, \ldots, 1)$ gibt es für $n \geq 4$?\\\\
        Wir sollen erstens ein Tupel mit 3 Elementen aus $n$ Elementen machen: 
        \[
        \binom{n}{3} = \frac{n!}{3!(n-3)!} 
        \textrm{\qquad(nach Satz 6. (ungeordnet, ohne Zur\"ucklegen) und Bin. Def.)}
        \]
        In dieser Tupel aus 3 Elementen haben wir 2 m\"ogliche Permutationen 
        (nach Satz 13., Zyklenschreibweise (Beispiel 3)). Folglich:
        \[\frac{n!}{3!(n-3)!} \cdot 2 = \frac{n!}{3\cdot(n-3)!}\]

        % Task (f)
        \item Kein Antwort.\\\\
        % Task (g)
        \item Zeigen Sie mittels vollständiger Induktion die Gleichung $S_{n,3} = \frac{1}{2} \cdot 3^{n-1} - 2^{n-1} + \frac{1}{2}$ für $n \ge 3$?\par
		(IA) $n=3:$ \[ S_{3,3} = 1 = \frac{1}{2} \cdot 3^{2} - 2^{2} + \frac{1}{2}\] \\
		(IV) \[S_{n-1,3} = \frac{1}{2} \cdot 3^{n-2} - 2^{n-2} + \frac{1}{2} \] \\
		(IS)             \begin{align*}
			S_{n,3} &= S_{n-1,2} + 3 \cdot S_{n-1,3} \tag{nach Theorem 1.15}\\
		            &= S_{n-1,2} + 3 \cdot (\frac{1}{2} \cdot 3^{n-2} - 2^{n-2} + \frac{1}{2} )  \tag{nach IV}\\
		            &= 2^{n-2} - 1 + \frac{3 \cdot 3^{n-2}}{2} -3 \cdot 2^{n-2} + \frac{3}{2} \tag{nach der Aufgabe f}\\
		            &= \frac{1}{2} 3^{n-1} - 2 \cdot 2^{n-2} -1 + 1\frac{1}{2} \\
		            &= \frac{1}{2} 3^{n-1} - 2^{n-1} + \frac{1}{2} 
		            		        	  \end{align*}
        % Task (h)
        \item Wie viele Möglichkeiten gibt es, eine ganzzahlige Zahl $n \ge 3k$ als Summe von $k$ ganzzahligen Summanden darzustellen, wobei jeder Summand mindestens $3$ ist?\\\\
        Nach Theorem 1.17: 
        \[ \binom{n-2k-1}{k-1} \]

        % Task (i)
        \item Kein Antwort.\\\\

        % Task (j)
        \item Wie viele Zahlen zwischen 1 und 100000 gibt es, deren Quersumme gerade 13 ergibt?\\\\
        Sei 13 die Summe von dreizehn 1. Dann k\"onnen wir diese 1 an 5 Positionen 
        stellen um eine Zahl, die Quersumme 13 hat, bekommen.\\
        Um 1 an 5 Gruppen zu Teilen k\"onnen wir solche Folge schreiben:
        $1...1 s s s s$, wo $s$ - eine Rolle von Separator spielt.\\ Dann z.B. diese
        Kombination $11111s111s1111s1$ bedeutet 5341 und diese $11111s111s11111s$
        bedeutet 5350. In diesem Fall benutzen wir Formel (siehe Blatt 2, Aufgabe f)
        , wo 1 wiederholt sich 13-Mal, Separator 4-Mal, und isngesamt es gibt 17 Elementen:
        \[\frac{n!}{k_1!\cdot k_2!\cdot\ldots\cdot k_n!} = \frac{17!}{13!\cdot4!} = 2380\]
        2380 enthalt aber auch Kombinationen, wann z.B. alle 1 in eine Gruppe sind. 
        Das passt uns naturlich nicht, da fuer jeder Ziffer maximal 9 Einsen m\"oglich sind.
        Deswegen sollen wir solche Faelle ausschliessen. Dann rechnen wir Zahl der 
        Kombinationen, wann 10,11,12,13 Eins in eine Gruppe kommen. \\
        Z.B fuer 10 haben wir noch 3 Einsen, die wir gruppieren sollen. Dann haben wir 111sss.
        \begin{align*}
        	&10: F_1 = 5 \cdot \frac{6!}{3!\cdot3!} = 100\\
        	&11: F_2 = 5 \cdot \frac{5!}{3!\cdot2!} = 50\\
        	&12: F_3 = 5 \cdot \frac{4!}{1!\cdot3!} = 20\\
        	&13: F_4 = 5 \cdot \frac{4!}{0!\cdot4!} = 5\\
        \end{align*}
        Dann ziehen wir unm\"ogliche Variationen aus unsere Zahl der gesamte Kombinationen aus:
        \[2380 - (F_1+F_2+F_3+F_4) = 2380 - 175 = 2205\]
    \end{enumerate}
    \section*{Kreativitat:}
    Beweisen Sie durch kombinatorische Argumente folgende Aussagen für fallende Faktorielle:
    \begin{enumerate}[label=(\alph*)]
    \item Für alle $n \in \mathbb{N}_+$ und $k \in \mathbb{N}_+$ gilt 	
    $n^{\underline{k}} = k \cdot (n-1)^{\underline{k-1}} +(n-1)^{\underline{k}}$.\\
    Durch $(n-1)^{\underline{k}}$ bekommen wir eine Variation aus $n-1$ Elementen.
    Da aber ein Element Fehlt, sollen wir eine Variation mit fehlenden Element machen.
    Dann nehmen wir ein Element weniger aus $n-1$ Elementen (um ein Platz fuer ein Element) 
    zu "reservieren" und multiplizieren mit Anzahl der Elementen (Stellen, wo wir
    fehlenden Element stehen lassen Koennen). \\\\
    Machen wir mal ein Beispiel.\\
    Wir haben ein Stundenplan aus $n-1 = 9$ Fachern und $k-1 = 5$ Lerhstunden gemacht:
    \[(n-1)^{\underline{k-1}}\]
    Dann wollten wir aber noch ein Fach und Lehrstunde hinzufuegen. Da wir koennen
    dieser mit 6 multiplizieren, da wir sechster Fach an 6 Stellen stellen koennen: 
    \[6\cdot(n-1)^{\underline{k-1}} \textrm{\qquad (1)}\]
    In diesem Fall aber haben wir alle Kombinationen, die neuer Fach enthalten. 
    Da ist aber auch moeglich, dass die manche Stundenplan Kombinationen neuer Fach 
    nicht enthalten sollen. Deswegen sollen wir noch ein Variation aus 9 Fachern 
    in 6 Lehrstunden machen: 
    \[(n-1)^{\underline{k}} \textrm{\qquad (2)}\]
    Folglich die Summe von Aussagen (1) und (2) ist eine Variation aus $n$ Fachern 
    an $k$ Lerhstunden. 
        \end{enumerate}
    \section*{Transfer:}
    \begin{enumerate}[label=(\alph*)]
\item Sei Punkte Team 1 = n und Punkte Team 2 = m.\\
		Konstruiere zunächst alle möglichen abfolgen um auf einen Punktestand n oder m zu kommen:
		$$\sum\limits_{k=0}^{\lfloor \frac{n}{2} \rfloor}\binom{n-k}{k}$$\\
		Die Summe zählt alle Möglichkeiten zusammen $2 \cdot k$ Einser Würfe durch k Zweier zu ersetzen. 
		\\
		Nun müssen die beiden Folge miteinander verknüpft werden.
		Hierzu schließen wir die beiden Listen aneinander an und verteilen in der Anzahl der Folgeglieder beider Listen die darin die zu werfenden 2er von Teamv1, die zu werfenden 2er von Team 2 und die zu werfenden Einser von Team 1. Die restlichen Einser von Team 2 müssen nicht mehr verteilt werden, da sich diese ergeben.
		
		$$ \sum\limits_{k=0}^{\lfloor \frac{n}{2} \rfloor} \sum\limits_{l=0}^{\lfloor\frac{m}{2}\rfloor}\binom{n-k+m-l}{k}\binom{n-2k+m-l}{l}\binom{n-2k+m-2l}{n-2k}$$
    \end{enumerate}
\end{document}