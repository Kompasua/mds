% Math symbols and examples: http://en.wikibooks.org/wiki/LaTeX/Mathematics
% Also useful link: http://en.wikibooks.org/wiki/LaTeX/Advanced_Mathematics

% This document is used for title generating of exercise solutions. 
% Also it used for generating styles and environmentof documents.
% To add this title in document add next line:
% % This document is used for title generating of exercise solutions. 
% Also it used for generating styles and environmentof documents.
% To add this title in document add next line:
% % This document is used for title generating of exercise solutions. 
% Also it used for generating styles and environmentof documents.
% To add this title in document add next line:
% \input{../template.tex}
\documentclass[a4paper]{article}
\usepackage[utf8]{inputenc}
\usepackage[top=1.5cm]{geometry}
\usepackage{amssymb}
\usepackage{enumitem}
\usepackage{amsmath}
\usepackage{fancyhdr}


\allowdisplaybreaks

\DeclareMathOperator{\ggT}{ggT}
\DeclareMathOperator{\kgV}{kgV}

\pagestyle{fancy}
\fancyhf{}
\rhead{Anton Bubnov, Yevgen Kuzmenko}
\lhead{Mathematik: Diskrete Strukturen}
\cfoot{\thepage}

\title{Mathematik: Diskrete Strukturen \\ \Large Lösungsblatt}
\author{Anton Bubnov, Yevgen Kuzmenko}

\documentclass[a4paper]{article}
\usepackage[utf8]{inputenc}
\usepackage[top=1.5cm]{geometry}
\usepackage{amssymb}
\usepackage{enumitem}
\usepackage{amsmath}
\usepackage{fancyhdr}


\allowdisplaybreaks

\DeclareMathOperator{\ggT}{ggT}
\DeclareMathOperator{\kgV}{kgV}

\pagestyle{fancy}
\fancyhf{}
\rhead{Anton Bubnov, Yevgen Kuzmenko}
\lhead{Mathematik: Diskrete Strukturen}
\cfoot{\thepage}

\title{Mathematik: Diskrete Strukturen \\ \Large Lösungsblatt}
\author{Anton Bubnov, Yevgen Kuzmenko}

\documentclass[a4paper]{article}
\usepackage[utf8]{inputenc}
\usepackage[top=1.5cm]{geometry}
\usepackage{amssymb}
\usepackage{enumitem}
\usepackage{amsmath}
\usepackage{fancyhdr}


\allowdisplaybreaks

\DeclareMathOperator{\ggT}{ggT}
\DeclareMathOperator{\kgV}{kgV}

\pagestyle{fancy}
\fancyhf{}
\rhead{Anton Bubnov, Yevgen Kuzmenko}
\lhead{Mathematik: Diskrete Strukturen}
\cfoot{\thepage}

\title{Mathematik: Diskrete Strukturen \\ \Large Lösungsblatt}
\author{Anton Bubnov, Yevgen Kuzmenko}


\begin{document}
    \maketitle
    \section*{Vertiefung:}
    \begin{enumerate}[label=(\alph*)]
        % Task (a)
        \item Welche Potenzreihe (in möglichst einfacher Darstellung mit nur einem Summensymbol) entspricht dem Produkt der Potenzreihen $\displaystyle\sum_{n=0}^{\infty} x^n$ und $\displaystyle\sum_{n=0}^{\infty} (2n+1)\cdot x^n$?\\
        \begin{align*}
            \sum_{n=0}^{\infty} x^n \cdot \sum_{n=0}^{\infty} (2n+1)\cdot x^n &= 
            \sum_{n=0}^{\infty} \Bigg(\sum_{k=0}^{n} (2n+1)\Bigg) \cdot x^n &\textrm{(Potenzreihen Rechnenregeln)}\\
            &= \sum_{n=0}^{\infty} (n+1)^2 \cdot x^n &\textrm{(BK Proposition 3.B)}
        \end{align*}

        % Task (b)
        \item Keine Antwort

        % Task (c)
        \item Keine Antwort

        % Task (d)
        \item Keine Antwort

        % Task (e)
        \item Keine Antwort

        % Task (f)
        \item Welche erzeugende Funktion hat die Folge $(-1, 1, -1, 1, -1, 1, \ldots , -1, 1, \ldots )$?\\
        \[\sum_{n=1}^{k} (-1)^n\]

        % Task (g)
        \item Keine Antwort

        % Task (h)
        \item Keine Antwort
    \end{enumerate}
    \section*{Kreativität:}
    \begin{enumerate}[label=(\alph*)]
        %Task (a)
        \item Task a
    \end{enumerate}
    \section*{Transfer:}
    \begin{enumerate}[label=(\alph*)]
        \item Task a
    \end{enumerate}
\end{document}






