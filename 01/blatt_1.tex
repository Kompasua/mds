% Math symbols and examples: http://en.wikibooks.org/wiki/LaTeX/Mathematics
% Also useful link: http://en.wikibooks.org/wiki/LaTeX/Advanced_Mathematics

% This document is used for title generating of exercise solutions. 
% Also it used for generating styles and environmentof documents.
% To add this title in document add next line:
% % This document is used for title generating of exercise solutions. 
% Also it used for generating styles and environmentof documents.
% To add this title in document add next line:
% % This document is used for title generating of exercise solutions. 
% Also it used for generating styles and environmentof documents.
% To add this title in document add next line:
% \input{../template.tex}
\documentclass[a4paper]{article}
\usepackage[utf8]{inputenc}
\usepackage[top=1.5cm]{geometry}
\usepackage{amssymb}
\usepackage{enumitem}
\usepackage{amsmath}

\allowdisplaybreaks

\DeclareMathOperator{\ggT}{ggT}
\DeclareMathOperator{\kgV}{kgV}

\title{Mathematik: Diskrete Strukturen \\ \Large Lösungsblatt}
\author{Anton Bubnov, Eugen Kuzmenko}

\documentclass[a4paper]{article}
\usepackage[utf8]{inputenc}
\usepackage[top=1.5cm]{geometry}
\usepackage{amssymb}
\usepackage{enumitem}
\usepackage{amsmath}

\allowdisplaybreaks

\DeclareMathOperator{\ggT}{ggT}
\DeclareMathOperator{\kgV}{kgV}

\title{Mathematik: Diskrete Strukturen \\ \Large Lösungsblatt}
\author{Anton Bubnov, Eugen Kuzmenko}

\documentclass[a4paper]{article}
\usepackage[utf8]{inputenc}
\usepackage[top=1.5cm]{geometry}
\usepackage{amssymb}
\usepackage{enumitem}
\usepackage{amsmath}

\allowdisplaybreaks

\DeclareMathOperator{\ggT}{ggT}
\DeclareMathOperator{\kgV}{kgV}

\title{Mathematik: Diskrete Strukturen \\ \Large Lösungsblatt}
\author{Anton Bubnov, Eugen Kuzmenko}


\begin{document}
	\maketitle
	\section*{Vertiefung:}
	\begin{enumerate}[label=(\alph*)]
		% Task (a)
		\item  Bestimmen Sie $ \bmod(5^{31} \cdot 2^{789}-23^{23}, 10) $. \\
		\textbf{Bemerkung:} die h\"ohere Potenzen werden in Potenzen von 2 ausgeteilt.\\ Ein Beispiel f\"ur $2^{2^n}$:
		\begin{align*}
			&\bmod(2,10)=2\\
			&\bmod(2^2,10)=4\\
			&\bmod(2^4,10)=\bmod(2^2\cdot2^2,10)=\bmod(\bmod(4,10)\cdot\bmod(4,10),10)=\bmod(16,10)=6\\
			&\bmod(2^8,10)=\bmod(2^4\cdot2^4,10)=\bmod(6\cdot6,10)=6\\
			&\bmod(2^{16},10)=\bmod(2^8\cdot2^8,10)=\bmod(6\cdot6,10)=6\\
			&\cdots \\
			&\bmod(2^{512},10)=\bmod(2^{256}\cdot2^{256},10)=\bmod(6\cdot6,10)=6
		\end{align*}
		Also auf solche Weise kann man rekursiv Modulus f\"ur Zahlen mit h\"ohe Potenzen berechnen.\\
		F\"ur Modulus Rechenregeln wird Theorem 1.2 (BM) benutzt. 
		\begin{align*}
			% part 1
			\bmod(5^{31},10) &= 5  \textrm{\qquad (da} \bmod(5^n,10)=5, \forall n \in \mathbb{N}, n\geq1) \\\\
			% part 2
			\bmod(2^{789},10) &= \bmod(2^{512}\cdot2^{256}\cdot2^{16}\cdot2^4\cdot2^1,10) 
			\\&= \bmod(\bmod(2^{512},10)\cdot\bmod(2^{256},10)\cdot\bmod(2^{16},10)\cdot\bmod(2^4,10)\cdot\bmod(2^1,10),10)
			\\&= \bmod(6\cdot 6\cdot 6\cdot 6\cdot 2, 10)
			\\&= \bmod(6\cdot2,10) 
			\\&= 2 \\\\
			% part 3
			\bmod(-23^{23},10) &= \bmod(23^{16}\cdot23^{4}\cdot23^{2}\cdot(-23)^1,10) 
			\\&= \bmod(\bmod(23^{16},10)\cdot\bmod(23^{4},10)\cdot\bmod(23^{2},10)\cdot\bmod((-23)^1,10),10)
			\\&= \bmod(9\cdot 7, 10) \textrm{\qquad (da} \bmod(23^2,10)=9 \textrm{ und }\bmod(23^4,10)=1)
			\\&= \bmod(63,10) 
			\\&= 3 \\\\
			% result
			\bmod(5^{31} \cdot 2^{789}-23^{23}, 10) &= \bmod(5 \cdot 2 + 3,10)	
			\\&= \bmod(13,10) 
			\\&= 3
		\end{align*}
\\
\\
\\
		% Task (b)
		\item  Bestimmen Sie $ \bmod(5^{31} \cdot 2^{789}-23^{23}, 11) $. \
		\begin{align*}
			% part 1
			\\ \bmod(5^{1},11) &= 5
			\\ \bmod(5^{2},11) &= \bmod(5 \cdot 5, 11) = 3
			\\ \bmod(5^{3},11) &= \bmod(3 \cdot 5, 11) = 4
			\\ \bmod(5^{4},11) &= \bmod(3 \cdot 3, 11) = 9
			\\ \bmod(5^{5},11) &= \bmod(3 \cdot 4, 11) = 1
			\\ \bmod(5^{31},11) &= \bmod(\bmod(5^5,11)^6 \cdot \bmod(5,11), 11)  \\
			&=5;
			\\\\
			% part 2
			\bmod(2^1,11) &= 2 \\
			\bmod(2^2,11) &= 4 \\
			\bmod(2^3,11) &= 8 \\
			\bmod(2^4,11) &= 5 \\
			\bmod(2^5,11) &= 10 \\
			\bmod(2^6,11) &= 9 \\
			\bmod(2^7,11) &= 7 \\
			\bmod(2^8,11) &= 3 \\
			\bmod(2^9,11) &= 6 \\
			\bmod(2^{10},11) &= 1 \\
			\bmod(2^{789},11) &= \bmod(\bmod(2^{10},11)^{78} \cdot \bmod(2^9,11),11) \\
			&= 6;
			\\\\
			% part 3
			\bmod(-23^{1},11) &=10\\
			\bmod(-23^2,11)&= 1 \\
			\bmod(-23^{23},11) &= \bmod(\bmod(-23^2,11)^{11} \cdot \bmod(-23,11),11) \\ 
			&= 10;
			\\\\
			% result
			\bmod(5^{31} \cdot 2^{789}-23^{23}, 11) &= \bmod(5 \cdot 6 +10,11)
			\\&= \bmod(40,11)
			\\&= 7 
		\end{align*}

		% Task (c)
		\item  Bestimmen Sie $ \bmod(7^{31} \cdot 2^{789}, 10) $. \\
		\textbf{Bemerkung:} Rechenweg ist gleich wie in Punkt (a).
		\begin{align*}
			% part 1
			\bmod(7^{31},10) &= \bmod(7^{16}\cdot7^{8}\cdot7^{4}\cdot7^{2}\cdot7^{1},10)
			\\&= \bmod(1\cdot1\cdot1\cdot9\cdot7,10)
			\\&= \bmod(63,10)
			\\&= 3 \\\\
			% part 2
			\bmod(2^{789},10) &= \bmod(2^{512}\cdot2^{256}\cdot2^{16}\cdot2^4\cdot2^1,10)
			\\&= \bmod(6\cdot 6\cdot 6\cdot 6\cdot 2, 10)
			\\&= \bmod(12,10)
			\\&= 2 \\\\
			% result
			\bmod(7^{31} \cdot 2^{789}, 10) &= \bmod(3 \cdot 2,10)
			\\&= \bmod(6,10)
			\\&= 6 
		\end{align*}

		% Task (d)
		\item Bestimmen Sie kgV(178, 144).
		\begin{align*}
			& 178 = 2\cdot89 \\
			& 144 = 2^4\cdot3^2 \\
			&\kgV(178, 144) = 2^4\cdot3^2\cdot89 = 12816
			% comment
		 	&\textrm{(nach Lemma 1.5 (BM))}
		\end{align*}

		% Task (e)
		\item Bestimmen Sie ggT(12877480, 24145275).
		\begin{align*}
			\ggT(12877480, 24145275) &= \ggT(24145275-12877480, 12877480)
			\\& = \ggT(12877480-11267795,11267795)
			\\& = \ggT(11267795-1609685,1609685)
			\\& = 1609685
			% comment
		 	&\textrm{(nach Lemma 1.8 (BM))}
		\end{align*}

		% Task (f)
		\item Wie sieht der $\frac{12877480}{24145275}$ zu äquivalente teilerfremde Bruch aus?
		\begin{align*}
			& \frac{12877480}{24145275} = \frac{2^3 \cdot 5 \cdot 7 \cdot 11 \cdot 37 \cdot 113}
			{3 \cdot 5^2 \cdot 7 \cdot 11 \cdot 37 \cdot 113} = \frac{2^3}{3 \cdot 5}
		\end{align*}

		% Task (g)
		\item Wie viele Funktionen $f : \{0, 1, 2, 3\}^n \to \{0, 1, 2\}$ gibt es, die genau einmal den Funktionswert 0 annehmen?
		\begin{enumerate}
			\item $\|\{0,1,2,3\}^n\| = 4^n =_{def}m$
			\item $\|\{0,1,2\}\| = 3 =_{def}n$
		\end{enumerate}
		Um 0 zu Stellen haben wir $m$ Moeglichkeiten. Andere Zahlen, also $n-1$, koennen wir
		an $m-1$ Stellen zu stellen. Also mit so viel Kombinationen: $(n-1)^{m-1}$
		Dann bekommen wir: 
		\[m\cdot(n-1)^{m-1} = 4^n\cdot(3-1)^{4^n-1} = 4^n\cdot3^{4^n-1} = 2^{4^n+2\cdot n-1}\]

		% Task (h)
		\item Wie viele Funktionen $f : \{0, 1, 2, 3\}^n \to \{0, 1, 2\}$ gibt es, die genau zweimal den Funktionswert 0 annehmen? \\
		\textbf{Bemerkung:} Definitionen (a) und (b) aus die Aufgabe (g) benutzt werden. \\
		Um erster 0 zu Stellen haben wir $m$ Moeglichkeiten. Fuer zweiter haben wir $m-1$.
		Also fuer zwei 0: $m\cdot (m-1)$. Andere Zahlen, also $n-1$, koennen wir
		an $m-2$ Stellen zu stellen. Also mit so viel Kombinationen: $(n-1)^{m-2}$
		Dann bekommen wir:
		\[m\cdot(m-1)\cdot(n-1)^{m-2} = 4^n\cdot(4^n-1)\cdot2^{4^n-1}\] 

		% Task (i)
		\item Wie viele Funktionen $f : \{0, 1, 2, 3\}^n \to \{0, 1, 2\}$ gibt es, die genau os oft die Funktionswerte 0 und 1 annehmen? \\
		\textbf{Bemerkung:} Definitionen (a) und (b) aus die Aufgabe (g) benutzt werden. \\
		Schreiben wir eine Falle, wenn 0 und 1 gleich oft gar nicht zu Ergebnnisse kommen, wenn beide einmal, zweimal usw. kommen.
		\[ \frac{m}{2} \left\{ 
			\begin{array}{l l}
		   		0: \quad 1  & \\
		    	1: \quad m\cdot(m-1) & \\
		    	2: \quad m\cdot(m-1)\cdot(m-2)\cdot(m-3) & \\
		    		\vdots \\
		    	m/2: \quad m\cdot(m-1)\cdot(m-2)\cdot(m-3)\cdot \ldots \cdot(m-(m-1))& 
		  	\end{array} 
		\right.\]
		Dann wenn wir alle diese Faelle summieren - bekommen wir so viele Funktionen, die 0 und 1 genau so oft annehmen. Also:
		\[1 \cdot m\cdot(m-1) \cdot m\cdot(m-1)\cdot(m-2)\cdot(m-3)\cdot \ldots \cdot m\cdot(m-1)\cdot(m-2)\cdot(m-3)\cdot \ldots \cdot(m-(m-1))\]

		% Task (j)
		\item Nach Lemma 4. $\|\{ f|f:\{0,1\} \to A \}\| = n^2$. 
		Nach Korollar 5. $\mathcal{P}(A) = 2^n$. \\ Also haben wir: $\varphi : 2^n \to n^2$. Dann wieder nach Lemma 4. bekommen wir: \[n^{2^{2^n}} = n^{2^{n+1}}\]
	\end{enumerate} 
	
	\section*{Transfer:}
	\begin{enumerate}[label=(\alph*)]
		\item 8 Symbole:
		\begin{enumerate}[label=\bfseries Schritt \arabic*:]
			% Schritt 1
			\item Wir w\"ahlen beliebig zwei S\"atze. Um L\"ange 8 zu bekommen mu\"ssen wir insgesamt 4 W\"ortern haben,
			weil andere 4 Symbolen zus\"atzliche Zeichen sind.

			% Schritt 2
			\item Da es in Alphabet 26 Buchstaben gibt, k\"onnen wir von 4 kleine Buchstaben so viel Kombinationen machen:
			\begin{align*}
				26^4=456976
			\end{align*}

			% Schritt 3
			\item Wenn eine halbe der Buchstaben gr\"oss sein kann, dann verdoppelt unsere Zahl der Kombinationen:
			\begin{align*}
				(26)^4\cdot2=456976\cdot2=913952
			\end{align*}

			% Schritt 4
			\item In Reihe von 4 Buchstaben k\"onnen wir Ziffern in 7 Varianten stellen:
			\begin{align*}
				& X \in {0,1,...,9} \\
				& \{a|bcd\} \to \{Xa|Xbcd\},\{aX|bcdX\} \\
				& \{ab|cd\} \to \{Xab|Xcd\},\{aXb|cXd\}, \{abX|cdX\} \\
				& \{abc|d\} \to \{Xabc|Xd\},\{abcX|dX\}
			\end{align*}
			Erste und zweite Zahlen kann 10 Ziffer sein. Also haben wir so viele Varianten f\"ur 2 Verschiedene 
			Ziffern und 7 Satz Kombinazionen:
			\begin{align*}
				& 10\cdot10\cdot7 = 700 \\
				& 913952\cdot700 = 639766400
			\end{align*}
			\textbf{Bemerkung:} In dieser Schritt "eine Ziffer mit Bedeutung in die beiden Satzbereiche" gem\"ass 
			Beispiel "LK1psDsIH2MG" war Interpretiert so, dass die Ziffern m\"ussen "symmetrisch" in der Satz stehen
			(gleiche Stelle an beide Satzen haben). 
			% Schritt 5
			\item N\"achstens haben wir 3 M\"oglichkeiten 6 Symbolen zu stellen:
			\begin{align*}
				& 3\cdot6=18 \\
				& 639766400\cdot18 = 11515795200
			\end{align*}

			% Schritt 6
			\item Dann haben wir 22 Symbole am Anfang oder am Ende:
			\begin{align*}
				& 22\cdot2=44 \\
				& 11515795200\cdot44 = 506694988800
			\end{align*}
		\end{enumerate}
		\rule{14cm}{0.4pt} \\
		Da in diese Aufgabe \"andert sich nur der Zahl der Symbole, kann man dazu ein Formel vervenden. 
		Anzahl der Kombinationen in Schritt 4 kann man so berechnen:
			\[k = \Big(\frac{n}{2}\Big)^2+n-1 \text{, wo n ist Anzahl der Symbole in Passwort minus 4.}\]
		Dann gem\"as der Informaton aus Schritten in (a) 1-6 bauen wir solche Funktion:
			\[f(n) = 26^n\cdot2\cdot10\cdot10\cdot\Big(\Big(\frac{n}{2}\Big)^2+n-1\Big)\cdot18\cdot44\]
		\rule{14cm}{0.4pt} \\
		\item 10 Symbole: f(6) = 685051624857600
		\item 12 Symbole: f(8) = 760798761663283200
		\item 14 Symbole: f(10) = 760269510350821785600
		\item 16 Symbole: f(12) = 710449496554891463884800
	\end{enumerate}
\end{document}