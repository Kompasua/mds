% Math symbols and examples: http://en.wikibooks.org/wiki/LaTeX/Mathematics
% Also useful link: http://en.wikibooks.org/wiki/LaTeX/Advanced_Mathematics

% This document is used for title generating of exercise solutions. 
% Also it used for generating styles and environmentof documents.
% To add this title in document add next line:
% % This document is used for title generating of exercise solutions. 
% Also it used for generating styles and environmentof documents.
% To add this title in document add next line:
% % This document is used for title generating of exercise solutions. 
% Also it used for generating styles and environmentof documents.
% To add this title in document add next line:
% \input{../template.tex}
\documentclass[a4paper]{article}
\usepackage[utf8]{inputenc}
\usepackage[top=1.5cm]{geometry}
\usepackage{amssymb}
\usepackage{enumitem}
\usepackage{amsmath}
\usepackage{fancyhdr}


\allowdisplaybreaks

\DeclareMathOperator{\ggT}{ggT}
\DeclareMathOperator{\kgV}{kgV}

\pagestyle{fancy}
\fancyhf{}
\rhead{Anton Bubnov, Yevgen Kuzmenko}
\lhead{Mathematik: Diskrete Strukturen}
\cfoot{\thepage}

\title{Mathematik: Diskrete Strukturen \\ \Large Lösungsblatt}
\author{Anton Bubnov, Yevgen Kuzmenko}

\documentclass[a4paper]{article}
\usepackage[utf8]{inputenc}
\usepackage[top=1.5cm]{geometry}
\usepackage{amssymb}
\usepackage{enumitem}
\usepackage{amsmath}
\usepackage{fancyhdr}


\allowdisplaybreaks

\DeclareMathOperator{\ggT}{ggT}
\DeclareMathOperator{\kgV}{kgV}

\pagestyle{fancy}
\fancyhf{}
\rhead{Anton Bubnov, Yevgen Kuzmenko}
\lhead{Mathematik: Diskrete Strukturen}
\cfoot{\thepage}

\title{Mathematik: Diskrete Strukturen \\ \Large Lösungsblatt}
\author{Anton Bubnov, Yevgen Kuzmenko}

\documentclass[a4paper]{article}
\usepackage[utf8]{inputenc}
\usepackage[top=1.5cm]{geometry}
\usepackage{amssymb}
\usepackage{enumitem}
\usepackage{amsmath}
\usepackage{fancyhdr}


\allowdisplaybreaks

\DeclareMathOperator{\ggT}{ggT}
\DeclareMathOperator{\kgV}{kgV}

\pagestyle{fancy}
\fancyhf{}
\rhead{Anton Bubnov, Yevgen Kuzmenko}
\lhead{Mathematik: Diskrete Strukturen}
\cfoot{\thepage}

\title{Mathematik: Diskrete Strukturen \\ \Large Lösungsblatt}
\author{Anton Bubnov, Yevgen Kuzmenko}

\DeclareMathOperator{\kgV}{kgV}
\usepackage{amsmath}
\usepackage{amssymb}

\begin{document}
    \maketitle
    \section*{Vertiefung:}
    \begin{enumerate}[label=(\alph*)]
        % Task (a)
        \item Ist $\langle \mathbb{R}_{>0} , f \rangle$ mit der Verknüpfung $f : (x,y) \mapsto  \log_y x$ eine Algebra? \\
        Ja, denn $\mathbb{R}_{>0}$ stellt eine nichtleere Menge dar. Der Operator $f$ ist zudem eine Abbildung, 
        der Logarithmus, der für alle $\mathbb{R}_{>0}$ gültig ist. Definition 4.1 ist somit erfüllt.
        
        % Task (b)
        \item Ist $\langle \{n| n \in \mathbb{N} \textrm{ ist eine ungerade Zahl} \}, f \rangle$ mit dem dreistelligen 
        Operator $f : (x,y,z) \mapsto \bmod(x^2 + 2y, z + 2)$ eine Algebra? \\
        Nein. Betrachte hierzu den Fall $x = 1, y = 1, z = 1$. In diesem Fall würde gelten 
        $$\bmod(1^2 + 2 \cdot 1, 1 + 2 ) = \bmod(3,3) = 0$$ Die Null ist aber eine gerade Zahl und muss auch 
        in $n$ liegen. Aus diesem Widerspruch folgt, dass hier keine Algebra vorliegt.
        
        % Task (c)
        \item
         Ist$\langle \mathbb{Z}_8 \setminus \{0\}, \cdot_k \rangle $ eine Algebra – und wenn nein, warum nicht? \\
         Nach dem Beispiel auf Seite 68 ist die Multiplikation $\cdot_k$ auf der Menge $\mathbb{Z}_k := \{0, 1, ..., k-1\}$ definiert. Für $a \cdot_8 b = mod(a \cdot b ,8) $ mit $ a = 2, b = 4 $ gilt $mod(2\cdot 4, 8) = 0$. Es gilt aber $0 \not \in \mathbb{Z}_8 \setminus {0}.$ Folglich kann hier keine Algebra definiert sein.
        % Task (d)
        \item
        Bestimmen Sie eine Teilmenge $A \subseteq \mathbb{Z}_{10} \setminus \{0\} $ mit maximaler Anzahl von Elementen, sodass$\langle A, \cdot_{10} \rangle$ eine Algebra ist. 
        $\langle \mathbb{Z} \setminus \{10, 5\}, \cdot_{10} \rangle $ ist eine Algebra mit maximaler Anzahl von Elementen, da kein  Produkt mehr durch 10 oder 5 teilbar ist.
        
        % Task (e)
        \item 
        Das neutrale Element ist 1.
        Da die Algebra $\langle\mathbb{N}_+; kgV \rangle$ kommutativ ist, reicht die Linksneutralität von 1. Sei $x \in \mathbb{ N}_+, n \setminus \{1\}$ beliebig aber fest.  Da 1 kein Vielfaches von x ist, gilt also $kgV (1, x) = 1 \cdot x =  x$. Da außerdem gilt $kgV (1, 1) = 1$, folgt nun, dass das neutrale Element 1 ist.
        
        % Task (f)
        \item Wie viele inverse Elemente besitzt 2 in der Algebra $\langle \mathbb{N}_+ , \kgV \rangle$?\\
        Überhaupt kein einziges, denn nach Definition 4.5 muss eine Inverse $x \in \mathbb{N}_+ $ mit $kgV(x,2) = 1$ existieren, aber der kgV von irgendwas und der 2 ist nicht 1 sondern 2. Aus diesem Widerspruch folgt, dass kein einziges inverses Element existieren kann.
        
        % Task (g)
        \item 
         Ist der Operator kgV assoziativ für die Algebra $\langle \mathbb{N}_+; kgV \rangle$ – und wenn nein, warum
         nicht?
         Der Operator kgV ist für die beschriebene Algebra assoziativ.\\
         Beweis: \\
         Seien $x, y, z \in \mathbb{N}_{+}$ beliebig und $\circ$ mit $\circ : (x,y) \mapsto kgv(x,y)$ definiert. Es folgt:\\
         \begin{align*} 
         	(x \circ y) \circ z &= \prod_{i=1}^{k} p_i^{\max(x_i,y_i)} \circ z \\
         	&=\prod_{i=1}^{k} p_i^{\max(\max(x_i,y_i),z_i)} \\
         	&=\prod_{i=1}^{k} p_i^{\max(x_i \max(y_i,z_i))} \\
         	&=x \circ \prod_{i=1}^{k} p_i^{\max(y_i,z_i)} \\
         	& x \circ (y \circ z)  
         \end{align*} 
         
        % Task (h)
        \item Können Sie die gegebene Verknüpfungstabelle für $\circ$ so ergänzen, dass $\circ$ assoziativ 
        auf \{a, b, c, d\} ist?\\
        Das geht nicht auf weil $\circ$ nicht assoziativ ist. Aus der Tabelle würde folgendes gelten, wenn $\circ$ assoziativ wäre: $$d = d \circ d = (b \circ c) \circ a = b \circ (c \circ a) = b \circ b = b$$
        Da $d \neq b$ ist kann keine Assoziativität für $\circ$ definiert sein.
       
        % Task (i)
        \item
         Gibt es eine Verknüpfung $\circ$ für die Algebra $\langle \{1, \dots, n\}, \circ \rangle$ an, so dass jedes $k \in \{1,\dots,n\}$ genau $k-1$ linksinverse Elemente besitzt? - Und wenn ja, welche?\\
         Es gibt eine solche Verknüpfung.
         Betrachte hierzu die Verknüpfung $\tilde{\circ}$  mit
         \begin{align*}
         \tilde{\circ}: ~~ & (\{1,\dots , n\},\{1,\dots, n\}) \rightarrow  \{1,\dots, n\}\\
         & (x,y) \mapsto \begin{cases}
         y, & falls x=n  \\
        \lfloor{\frac{n+1}{2}} & falls x = 1 \land y =n\rfloor \\
        n, & falls y > x\\
        \lfloor{\frac{x+y}{2}}, & \text{sonst.}\rfloor \\
        \end{cases}
        \end{align*}
        Das linksneutrale Element ist $n$, denn $(n,y)=y$ gilt für beliebige $y$. Sei $k\ne n$ fest und beliebig und gelte $k > x_0$. Dann kann $x_0$ $k-1$ mögliche Werte annehmen. Zudem folgt aus $\tilde{\circ}$, das $(x_0,k)=n$ gilt. Für $k = n$ gilt $(1,n) \ne n$ und sonst $(x,n) = n$. Damit folgt, dass es eine solche Verknüpfung gibt.       
        
        % Task (j)
        \item         
         Gibt es eine Verknüpfung $\circ$ für die Algebra $\langle \{1, \dots, n\}, \circ \rangle$ an, so dass jedes $k \in \{1,\dots,n\}$ genau $k-1$ rechtsinverse Elemente besitzt? - Und wenn ja, welche?\\
         Es gibt eine solche Verknüpfung. Betrachte hierzu die Verknüpfung $\tilde{\circ}$ 
         \begin{align*}
         \tilde{\circ}: ~~ & (\{1,\dots , n\},\{1,\dots, n\}) \rightarrow  \{1,\dots, n\}\\
         & (x,y) \mapsto \begin{cases}
         x, & y=n  \\
         \lfloor{\frac{n+1}{2}}\rfloor, & x = n \land y =1\\
         n, & x > y\\
         \lfloor{\frac{x+y}{2}}\rfloor, & \text{sonst.}\\
         \end{cases}
         \end{align*}
         Das rechtsneutrale Element ist $n$, da $(x,n)=x$ ($x$ beliebig) gilt. Sei $k\ne n$ fest und beliebig und gelte $k > y_0$. Dann kann $y_0$ $k-1$ mögliche Werte annehmen. Zudem folgt aus $\tilde{\circ}$, das $(k,y_0)=n$ gilt. Für $k = n$ gilt $(n,1) \ne n$ und und sonst $(n,y) = n$. Damit folgt die Behauptung.
    \end{enumerate}
    \section*{Kreativität:}
    \begin{enumerate}[label=(\alph*)]
    	%Task (a)
    	\item %sorry, ich glaube das geht generell nicht...
    \end{enumerate}
    \clearpage
    \section*{Transfer:}
    \begin{enumerate}[label=(\alph*)]
    	\item
        \begin{align*}
            (R \Join S) \Join T &= \{(a,d) \, | \, \exists\, c \in C : (a,c) \in R \Join S \land (c,d) \in T \}\\
            &=\{(a,d) \, | \, \exists\, c \in C : (a,c) \in \{(a,c) \,|\, \exists\, b \in B : (a,b) \in R \land (b,c) 
            \in S \}  \land (c,d) \in T \}\\
            &=\{(a,d) \, | \, \exists\, c \in C \, \exists\, b \in B : (a,b) \in R \land (b,c) \in S   \land (c,d) \in T \}\\
            &=\{(a,d) \, |  \, \exists\, b \in B \, \exists\, c \in C :  (b,c) \in S   \land (c,d) \in T \land (a,b) 
            \in R  \}\\
            &=\{(a,d) \, |  \, \exists\, b \in B : (b,d) \in \{(b,d) \,|\, \exists\, c \in C :  (b,c) \in S   
            \land (c,d) \in T\} \land (a,b) \in R  \}\\
            &=\{(a,d) \, |  \, \exists\, b \in B : (b,d) \in S\Join T \land (a,b) \in R  \}\\
            &=\{(a,d) \, |  \, \exists\, b \in B : (a,b) \in R \land (b,d) \in S\Join T  \}\\
            &=R \Join (S \Join T)
        \end{align*}
        \item
        Die Kardinalitet aller Joins in der Tabelle ist 8 wenn wir so klammern:
        $$  \bigg( (E \Join E) \Join (E \Join E) \bigg) \Join W$$
        und nach der Regel 'innere Klammer zuerst' vorgehen. Dann wird $(E \Join E)$ nur einmal ausgefuert. 
        Aus der Definition des Joins (und nach der Vorlesung in Datenbanksysteme ) folgt schließlich, die Idempotenz für $\Join$, also $E \Join E = E$ und damit auch $||E \Join E|| = ||E||$. Somit gilt folgendes:\\
        Für die beiden kleinen inneren Klammern erhalten wir jeweils 8 Tupel aufgrund der Idempotenz.\\
        Für die große äußere Klammer erhalten wir auch wieder 8 Tupel aufgrund der Idempotenz.
        Für den letzten Join unserer errechneten 8 Tupel mit der Relation W erhalten wir ebenfalls wieder 8 Tupel in der Ergebnisrelation. 
        Da wir in jeder Ausführungsfolge 8 Tupel erhalten, betragen die Gesamtplatzkosten 
        $$3\cdot 8 = 24$$
        
        
    \end{enumerate}
\end{document}