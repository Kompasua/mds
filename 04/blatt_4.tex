% Math symbols and examples: http://en.wikibooks.org/wiki/LaTeX/Mathematics
% Also useful link: http://en.wikibooks.org/wiki/LaTeX/Advanced_Mathematics

% This document is used for title generating of exercise solutions. 
% Also it used for generating styles and environmentof documents.
% To add this title in document add next line:
% % This document is used for title generating of exercise solutions. 
% Also it used for generating styles and environmentof documents.
% To add this title in document add next line:
% % This document is used for title generating of exercise solutions. 
% Also it used for generating styles and environmentof documents.
% To add this title in document add next line:
% \input{../template.tex}
\documentclass[a4paper]{article}
\usepackage[utf8]{inputenc}
\usepackage[top=1.5cm]{geometry}
\usepackage{amssymb}
\usepackage{enumitem}
\usepackage{amsmath}

\allowdisplaybreaks

\DeclareMathOperator{\ggT}{ggT}
\DeclareMathOperator{\kgV}{kgV}

\title{Mathematik: Diskrete Strukturen \\ \Large Lösungsblatt}
\author{Anton Bubnov, Eugen Kuzmenko}

\documentclass[a4paper]{article}
\usepackage[utf8]{inputenc}
\usepackage[top=1.5cm]{geometry}
\usepackage{amssymb}
\usepackage{enumitem}
\usepackage{amsmath}

\allowdisplaybreaks

\DeclareMathOperator{\ggT}{ggT}
\DeclareMathOperator{\kgV}{kgV}

\title{Mathematik: Diskrete Strukturen \\ \Large Lösungsblatt}
\author{Anton Bubnov, Eugen Kuzmenko}

\documentclass[a4paper]{article}
\usepackage[utf8]{inputenc}
\usepackage[top=1.5cm]{geometry}
\usepackage{amssymb}
\usepackage{enumitem}
\usepackage{amsmath}

\allowdisplaybreaks

\DeclareMathOperator{\ggT}{ggT}
\DeclareMathOperator{\kgV}{kgV}

\title{Mathematik: Diskrete Strukturen \\ \Large Lösungsblatt}
\author{Anton Bubnov, Eugen Kuzmenko}

\DeclareRobustCommand{\stirling}{\genfrac\{\}{0pt}{}}
\begin{document}
    \maketitle
    \section*{Vertiefung:}
    \begin{enumerate}[label=(\alph*)]
        % Task (a)
        \item Drücken Sie die Anzahl der surjektiven Funktionen 
        $f : \{0, 1\}^n \to \{0, 1\}^2$ mit Hilfe der Stirling-Zahlen zweiter Art aus.\\\\
        Nach Lemma 4 (Potenzregel) und Kreuzprodukt Definition es gilt entsprechend:
        \begin{align*}
	        &||\{0, 1\}^n|| = 2^n \\
	        &|| \{0, 1\}^2|| = 2^2 = 4
	    \end{align*}
	    Wir muessen $2^n$ Funktionsargumente auf $4$ Funktionswerte abbilden. Da Stirling-Zahlen auf 
	    nicht unterscheidbare Funktionswerte aufzuteilt, sollen wir noch nit $4!$ multiplizieren.  
	    Folglich: \[4!\cdot\stirling{a}{b}\]

        % Task (b)
        \item Kein Antwort

        % Task (c)
        \item Von 18 Studierenden in einer Spezialvorlesung studieren 7 Mathematik, 9 Physik und 10
		Informatik. Davon studieren 3 Mathematik und Physik, 3 Mathematik und Informatik
		sowie 5 Physik und Informatik. Ein Student studiert sogar all drei Fächer. Wie viele
		Studierende studieren keines der drei Fächer?\\
		Sei nach Voraussetzung:
		\[||M|| = 7,\, ||I|| = 10,\, ||P|| = 9,\, ||M \cap P|| = 3,\, ||M \cap I|| = 3,\, 
		  ||P \cap I|| = 5,\, ||M \cap P \cap I|| = 1\]
		Die gesamte Zahl der Studierenden, die in einer Spezialvorlesung studieren ist:
		\[||M \cup P \cup I||\]
		Nach Theorem 1.19 Beispiel es gilt:
		\begin{align*}
	        ||M\cup P\cup I|| &= ||M||+||P||+||I||-||M\cap P||-||M\cap I||-||I \cap P||+||M \cap P \cap I||\\
	        &= 7 + 9 + 10 - 3 - 3 - 5 + 1 \\
	        &= 16
        \end{align*}
        Folglich die Anzahl der Studierende, die keines der drei Fächer studieren ist $18-16 = 2$.

        % Task (d)
        \item Für drei Mengen $A, B$ und $C$ gelten folgende Eigenschaften: 
        $||A|| = 63, ||B|| = 91, ||C|| = 44, ||A\cap B|| = 25, ||A\cap C|| = 23, ||C \cap B|| = 21$. 
        Außerdem gelte $||A\cup B\cup C|| = 139$. Wie groß ist $||A \cap B \cap C||$?\\\\
        Nach Theorem 1.19 Beispiel es gilt:
        \begin{align*}
	        ||A\cup B\cup C|| &= ||A||+||B||+||C||-||A\cap B||-||A\cap C||-||C \cap B||+||A \cap B \cap C||\\
	        \Rightarrow ||A \cap B \cap C|| &= ||A\cup B\cup C||-||A||-||B||-||C||+||A\cap B||+||A\cap C||+||C \cap B||\\
	        \Rightarrow ||A \cap B \cap C|| &= 139 - 63 - 91 - 44 + 25 + 23 + 21 = 10 \\
        \end{align*}

        % Task (e)
        \item Kein Atnwort

        % Task (f)
        \item Wie viele Zahlen im Bereich $1, 2,\ldots, 200$ sind durch keine der Zahlen 3, 7, 11, 27 teilbar?\\
        \[200 - \bigg(\bigg\lfloor\frac{200}{3}\bigg\rfloor + \bigg\lfloor\frac{200}{7}\bigg\rfloor
         + \bigg\lfloor\frac{200}{11}\bigg\rfloor + \bigg\lfloor\frac{200}{27}\bigg\rfloor \bigg)
         = 200 - (66+28+18+7) = 81
        \]
    \end{enumerate}
\end{document}