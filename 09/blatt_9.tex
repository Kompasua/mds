% Math symbols and examples: http://en.wikibooks.org/wiki/LaTeX/Mathematics
% Also useful link: http://en.wikibooks.org/wiki/LaTeX/Advanced_Mathematics

% This document is used for title generating of exercise solutions. 
% Also it used for generating styles and environmentof documents.
% To add this title in document add next line:
% % This document is used for title generating of exercise solutions. 
% Also it used for generating styles and environmentof documents.
% To add this title in document add next line:
% % This document is used for title generating of exercise solutions. 
% Also it used for generating styles and environmentof documents.
% To add this title in document add next line:
% \input{../template.tex}
\documentclass[a4paper]{article}
\usepackage[utf8]{inputenc}
\usepackage[top=1.5cm]{geometry}
\usepackage{amssymb}
\usepackage{enumitem}
\usepackage{amsmath}
\usepackage{fancyhdr}


\allowdisplaybreaks

\DeclareMathOperator{\ggT}{ggT}
\DeclareMathOperator{\kgV}{kgV}

\pagestyle{fancy}
\fancyhf{}
\rhead{Anton Bubnov, Yevgen Kuzmenko}
\lhead{Mathematik: Diskrete Strukturen}
\cfoot{\thepage}

\title{Mathematik: Diskrete Strukturen \\ \Large Lösungsblatt}
\author{Anton Bubnov, Yevgen Kuzmenko}

\documentclass[a4paper]{article}
\usepackage[utf8]{inputenc}
\usepackage[top=1.5cm]{geometry}
\usepackage{amssymb}
\usepackage{enumitem}
\usepackage{amsmath}
\usepackage{fancyhdr}


\allowdisplaybreaks

\DeclareMathOperator{\ggT}{ggT}
\DeclareMathOperator{\kgV}{kgV}

\pagestyle{fancy}
\fancyhf{}
\rhead{Anton Bubnov, Yevgen Kuzmenko}
\lhead{Mathematik: Diskrete Strukturen}
\cfoot{\thepage}

\title{Mathematik: Diskrete Strukturen \\ \Large Lösungsblatt}
\author{Anton Bubnov, Yevgen Kuzmenko}

\documentclass[a4paper]{article}
\usepackage[utf8]{inputenc}
\usepackage[top=1.5cm]{geometry}
\usepackage{amssymb}
\usepackage{enumitem}
\usepackage{amsmath}
\usepackage{fancyhdr}


\allowdisplaybreaks

\DeclareMathOperator{\ggT}{ggT}
\DeclareMathOperator{\kgV}{kgV}

\pagestyle{fancy}
\fancyhf{}
\rhead{Anton Bubnov, Yevgen Kuzmenko}
\lhead{Mathematik: Diskrete Strukturen}
\cfoot{\thepage}

\title{Mathematik: Diskrete Strukturen \\ \Large Lösungsblatt}
\author{Anton Bubnov, Yevgen Kuzmenko}

\usepackage{wrapfig}
\usepackage{color}

\begin{document}
    \maketitle
    \section*{Vertiefung:}
    \begin{enumerate}[label=(\alph*)]
        % Task (a)
        \item  Ein ebener, $k$-regulärer Graph besteht aus 12 Knoten und teilt die Ebene in 14 Gebiete. 
        Wie groß ist $k$? \\
        Nach Theorem 3.23 gilt $||F|| = ||E|| - ||V|| + 2$. Für unseren Fall bedeutet das also:
        $$14 = ||E|| - 12 + 2 \implies E = 24$$ Des weiteren gilt nach Proposition 3.3:
        $$\sum_{v \in V} deg(v) = 2 \cdot ||E||$$ D.h. die Summe der Grade ist $2 * 24 = 58$.
        Daraus folgt $\frac{58}{12} = 4 = k$
        
        %Task (b)
        \item Gibt es einen ebenen, $k$-regulären Graphen mit 8 Knoten, der die Ebene in 14 Gebiete
        teilt? \\  
        Aus Theorem 3.23 folgt, dass $||F|| = ||E|| - ||V|| + 2$ gilt. Für unseren Fall bedeutet 
        das aus $ 14 = ||E|| - 8 + 2 $ $ ||E|| = 20 $ folgt. Wir haben also einen Graphen mit 8 Knoten 
        und 20 Kanten. Nach Theorem 3.24 gilt für jeden planaren Graphen $ ||E|| \leq 3 \cdot ||V|| - 6$. 
        Für unseren Fall gilt also $20 \leq 3 \cdot 8 - 6  = 18$.
        Wir stoßen auf einen Widerspruch, da $20 \leq 18$ falsch ist. Einen solchen Graphen kann es nicht geben.
        
        %Task (c)
        \item Erweitern Sie die Eulersche Polyederformel auf nichtzusammenhängende, planare Gra-
        phen. \\
        Für einen planaren nicht-zusammenhängenden Graphen G mit k Zusammenhangskomponenten beachten 
        wir, dass sich die Komponenten die äußere Fläche gemeinsam teilen. Daraus folgt, dass es im gegensatz 
        zum zusammenhängenden Fall k-1 Flächen weniger gibt. Somit gilt für nicht zusammenhängende Graphen 
        $$||F|| = ||E|| - ||V|| + 2 + (k-1) = ||E|| - ||V|| + 1 + k$$ Für nicht-zusammenhängende Graphen 
        ergibt sich folglich: $$||F|| = ||E|| - ||V|| + 1 + k $$
       
        %Task (d)
        \item Ist der $Q_3$ planar? \\
        Ja. Betrachte dazu folgende Graphik:\\
        \includegraphics[width=0.3\linewidth]{Q3}
        
        %Task (e)
        \item Ist der $Q_4$ planar?\\
        Nein. Der $Q_4$ enthält einen $K_{3,3}$ als Teilgraphen. wenn man diesen Teilgraphen ''aufspannt'' 
        gibt es Überkreuzungen im gesamten Graphen. Daraus folgt, dass der $Q_4$ nicht planar sein kann. \\
        hier der (gut versteckte) $K_{3,3}$ mit \textcolor{magenta}{direkten Verbindungen} und \textcolor{green}{indirekten Verbindungen} \\ \includegraphics{Q4}
        
        %Task (f)
        \item Es reicht aus die Aussage für kantenmaximale planare Graphen zu zeigen, da hiermit die 
        linke Seite maximiert wird. Sei nun $G=(V,E)$ ein kantenmaximaler und ebener Graph, der keine 
        Dreiecke als Teilgraphen besitzt. Für $||V||=3$ gibt es nur das Äußere und somit nur ein Gebiet. 
        Wenn wir nun einen Knoten hinzufügen, dann müssen wir auch aufgrund der Kantenmaximalität zwei 
        Kanten hinzufügen, wodurch Dreiecke vermieden werden.  Wenn wir dies fortsetzen, erhalten wir:
        $$ 2 \cdot ||F|| \le ||E||\implies 2 \cdot \left(||E|| - ||V|| + 2 \right) \le ||E||$$
        Nach Theorem 3.23 gilt somit:
        $$||E|| \le 2 \cdot ||V|| - 4 $$ 
        
        % Task (g)
        \item Kein Antwort
        
        % Task (h)
        \item Kein Antwort

        % Task (i)
        \item Wie viele Knotenfärbungen mit k Farben hat ein $K_n$?
        Da jeder Knoten n-1 Nachbarn hat (jeder Knoten ist mit jedem verbunden) hat der $K_n$ $n$ Farben. 
        Der erste Knoten kann beliebig gewählt werden, weshalb es $k$ Möglichkeiten zur Färbung gibt. 
        Für die weiteren Knoten gibt es immer eine Möglichkeit weniger, für eine Färbung. Folglich gibt es: \\
        $\prod_{i=0}^{k-1} k - i$ Möglichkeiten zur Färbung eines $K_n$.
          
        % Task (j)
        \item Wie ist die chromatische Zahl eines $Q_d$?
        Nach Defifnition des Hyperwürfels gilt, dass der $Q_d$ immer quadratische Flächen hat bzw. Kreise der Länge 4. Da man für einen solchen Kreis minimal zwei Farben braucht, gibt es auch einen $Q_d$ der mit nur zwei Farben auskommt. Die chromatische
        
    \end{enumerate}
    \section*{Kreativität:} Wie viele Knotenfärbungen mit 3 Farben hat der Kreis $C_n$ ?\\
    Hinweis: Stellen Sie eine geeignete Rekursionsgleichung auf und lösen Sie diese.\\
    \\
    Wenn wir anfangen den Kreis mit unseren 3 Farben zu färben, dann können wir bei einem beliebigen Knoten unseren Anfang setzen. Wir denken uns nun eine der beiden Kanten vom Anfangsknoten weg und erhalten einen zusammenhängenden Graphen mit der Eigenschaft $||E|| = n-1$, also einen Baum. Um den Anfangsknoten zu färben stehen 3 Möglichkeiten zur Verfügung, wodurch sich 3 verschieden gefärbte Binärbäume konstruieren lassen, die aber alle die gleiche Struktur haben (es entstehen verschiedene Binärbäume da die Wurzelfarbe jeweils unterschiedlich ist). Zur Visualisierung zeigt die eingefügte Graphik den konstruierten Binärbaum bis hin zum $C_6$ für den Fall, dass man dem ersten Knoten die Farbe 1 zugeordnet hat. Offensichtlich gibt es im Kreisgraphen bis zum Vorletzten Knoten 2 Möglichkeiten zur weiteren Färbung. Wir halten im Binärbaum mit jeden weiteren Blättern die "Färbungspfade" fest. 
        \begin{wrapfigure}{l}{0.25\textwidth}
            \includegraphics[width=1\linewidth]{baum_kreativ}
        \end{wrapfigure}\\
     Wir können sehen, dass die eingefügte Graphik alle Variationen für einen $C_1$ bis hin zum $C_6$  wiedergibt. \\
     Wir müssen allerdings beachten, dass wir manche Knoten 'zu oft abzählen', da sie nicht in das Färbungsschema hineinpassen. In unserer Graphik sind das beispielsweise für den $C_6$  immer die Knoten, die die Farbe 1 haben. Wir müssen folglich für jeden der drei Bäume  mit den verschiedenen Wurzelfärbungen stets die Anzahl an Knoten abziehen, die eine Farbe erhalten würden, die mit der Farbe der Baumwurzel übereinstimmt.\\
     Für den Binärbaum, mit der Baumwurzel, die die Farbe 1 erhalten hat, ist die gesamte Anzahl der Knoten $2^n-1$. Wir färben nun jedes Blatt mit jeweils einer anderen Farbe wir die Wurzel gefärbt haben. Für alle weiteren Blätter bzw. Subbäume wenden wir das gleiche Färbungsschema an und erhalten somit einen Baum mit $2^n-1$ Färbungen. Da wir den ersten Knoten (die Baumwurzel) auch mit der 2 oder 3 Färben können, lassen sich analog 2 weitere Binärbäume aufspannen, die das gleiche Schema besitzen, aber weitere Möglichkeiten zur Färbung, die uns fehlen, wenn wir die Wurzel mit der 1 färben. Wir halten eine 'Gesamtfarbenbelegungsanzahlanzahl' mit $3\cdot 2^{n-1}$ Möglichkeiten (vorerst) fest. Dies ist gerade die Anzahl der Knoten in den untersten Baumebenen.
     Von dieser müssen wir nun die Anzahl der gefärbten Knoten abziehen, die uns nicht passen. Dies sind gerade diejenigen Knoten in der untersten Baumebene, deren Farbe mit der Farbe der Baumwurzel übereinstimmt. Die Anzahl dieser Knoten erhalten wir stets, indem wir uns die Möglichkeiten für die Färbung in der "Baumebene darüber" berechnen.
     
     Somit können wir folgende rekursive Formel aufstellen, die die Anzahl der Knotenfärbungen 
     berechnet. Für den $C_0$  und $C_1$ gibt es keine Möglichkeiten, denn ein Kreis muss aus mindestens 3 Knoten bestehen. Für die Gültigkeit unserer rekursiven Formel fassen wir fassen einen Pfad aus zwei Knoten und einer Kante als Sonderfall eines Kreises auf ( der 6 mögliche Färbungen besitzen kann) und für den $C_n$ gelte somit: 
     \begin{align*} 
     a_0 &= 0 \\ 
     a_1 &= 0  \\
     a_n &= 3\cdot 2^{n-1} - a_{n-1}\\ 
     \end{align*}
     Nun wollen wir unsere rekursive Formel in eine explizite (nicht-rekursive) Form überführen. Wir schreiben die ersten 5 Knotenfärbungen auf:
     \begin{align*}
          a_0 &= 0\\
          a_1 &= 0\\
          a_2 &= 3\cdot 2^1 - a_1 = 3\cdot 2^1\\
          a_3 &= 3\cdot 2^2 - a_2 = 3\cdot 2^2 - 3\cdot 2^1\\
          a_4 &= 3\cdot 2^3 - a_3 = 3\cdot 2^3 - (3\cdot 2^2 - 3\cdot 2^1)\\
          a_5 &= 3\cdot 2^4 - a_4 = 3\cdot 2^4 - (3\cdot 2^3 - (3\cdot 2^2 - 3\cdot 2^1))\\
          &= 3\cdot 2^4 - 3\cdot 2^3 + 3\cdot 2^2 - 3\cdot 2^1\\
          &= 3\cdot (2^4 - 2^3 + 2^2 - 2^1)
        \end{align*} 
     Ohne Beeinträchtigung der Allgemeinheit können wir folglich unsere rekursive Formel wie folgt überschreiben:
     $$ a_n = 3 \cdot (-1)^{n-1} \cdot \sum_{k=1}^{n-1}(-2)^k$$
    \section*{Transfer:}
    \begin{enumerate}[label=(\alph*)]
    	\item Wie viele Kanten benötigen Sie für einen (n, d)-dimensionalen De Bruijn-Graphen?\\
        Wir haben zwei Fällen. In einer Fall haben wir zwei gleiche Zahlen am Ende, da ist nur in $d$ 
        Kombinationen möglich ist. Dann für jeder dieser Zahl haben wir nur zwei $d-1$ Möglichkeiten. 
        Also insgesamt: $d\cdot(d-1)$ Kanten. Zweiten Fall ist alle andere Fälle, also $d^n-d$. Dann haben
        wir: $d\cdot(d^n-d)$. Folglich insgesamt es gibt so viel Kanten:
        $$d\cdot(d-1) + d\cdot(d^n-d) = d^{n+1}-d$$
    \end{enumerate}
\end{document}