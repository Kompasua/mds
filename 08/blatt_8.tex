% Math symbols and examples: http://en.wikibooks.org/wiki/LaTeX/Mathematics
% Also useful link: http://en.wikibooks.org/wiki/LaTeX/Advanced_Mathematics

% This document is used for title generating of exercise solutions. 
% Also it used for generating styles and environmentof documents.
% To add this title in document add next line:
% % This document is used for title generating of exercise solutions. 
% Also it used for generating styles and environmentof documents.
% To add this title in document add next line:
% % This document is used for title generating of exercise solutions. 
% Also it used for generating styles and environmentof documents.
% To add this title in document add next line:
% \input{../template.tex}
\documentclass[a4paper]{article}
\usepackage[utf8]{inputenc}
\usepackage[top=1.5cm]{geometry}
\usepackage{amssymb}
\usepackage{enumitem}
\usepackage{amsmath}

\allowdisplaybreaks

\DeclareMathOperator{\ggT}{ggT}
\DeclareMathOperator{\kgV}{kgV}

\title{Mathematik: Diskrete Strukturen \\ \Large Lösungsblatt}
\author{Anton Bubnov, Eugen Kuzmenko}

\documentclass[a4paper]{article}
\usepackage[utf8]{inputenc}
\usepackage[top=1.5cm]{geometry}
\usepackage{amssymb}
\usepackage{enumitem}
\usepackage{amsmath}

\allowdisplaybreaks

\DeclareMathOperator{\ggT}{ggT}
\DeclareMathOperator{\kgV}{kgV}

\title{Mathematik: Diskrete Strukturen \\ \Large Lösungsblatt}
\author{Anton Bubnov, Eugen Kuzmenko}

\documentclass[a4paper]{article}
\usepackage[utf8]{inputenc}
\usepackage[top=1.5cm]{geometry}
\usepackage{amssymb}
\usepackage{enumitem}
\usepackage{amsmath}

\allowdisplaybreaks

\DeclareMathOperator{\ggT}{ggT}
\DeclareMathOperator{\kgV}{kgV}

\title{Mathematik: Diskrete Strukturen \\ \Large Lösungsblatt}
\author{Anton Bubnov, Eugen Kuzmenko}

% Used to draw graphs. More info: http://en.wikipedia.org/wiki/PGF/TikZ
% Examples: http://www.texample.net/tikz/examples/
\usepackage{tikz} 

\begin{document}
    \maketitle
    \section*{Vertiefung:}
    \begin{enumerate}[label=(\alph*)]
        % Task (a)
        \item Welche Gradfolge besitzt $M_{3,4}$?\\
        Die Gradfolge von $M_{3,4}$ ist $4,4,3,3,3,3,3,3,2,2,2,2$.

        % Task (b)
        \item Gibt es einen Graphen mit Gradfolge $(5, 4, 3, 2, 2, 2, 2)$? - Und wenn ja, welchen?\\
        \begin{tikzpicture}
            [scale=.8,auto=left,every node/.style={circle,fill=blue!20}]
            \node (n1) at (3,1) {5};
            \node (n2) at (1,-1) {2};
            \node (n3) at (3,-1) {2};
            \node (n4) at (4,-1) {2};
            \node (n6) at (5,-1) {2};
            \node (n7) at (2,-2) {4};
            \node (n8) at (4,-2) {3};

            \foreach \from/\to in {n1/n2, n1/n3, n1/n4, n1/n6, n7/n1, n7/n2, n7/n3, n7/n8, n8/n6, n8/n4}
            \draw (\from) -- (\to);
        \end{tikzpicture}\\\\

        % Task (c)
        \item Gibt es einen Graphen mit Gradfolge $(5, 4, 3, 2, 2, 2, 2, 1)$? - Und wenn ja, welchen?\\
        Gibt es keinen.

        % Task (c)
        \item Gibt es einen Graphen mit Gradfolge $(5, 2, 1, 1, 1, 1, 1, 1, 1)$? - Und wenn ja, welchen?\\
        Gibt es keinen.\\\\
        \begin{tikzpicture}
            [scale=.8,auto=left,every node/.style={circle,fill=blue!20}]
            \node (n1) at (3,1) {5};
            \node (n2) at (1,-1) {1};
            \node (n3) at (2,-1) {1};
            \node (n4) at (3,-1) {1};
            \node (n5) at (4,-1) {1};
            \node (n6) at (5,-1) {1};
            \node (n7) at (6,-1) {1};
            \node (n8) at (7,-1) {1};

            \node (n9) at (6,1) {2};

            \foreach \from/\to in {n1/n2, n1/n3, n1/n4, n1/n5, n1/n6, n9/n8, n9/n7}
            \draw (\from) -- (\to);
        \end{tikzpicture}\\\\
    \end{enumerate}
    \section*{Kreativität:}
    \begin{enumerate}[label=(\alph*)]
        %Task (a)
        \item Task a
    \end{enumerate}
    \section*{Selbststudium:}
        Erarbeiten Sie sich den Inhalt des Abschnitts "Bäume und Wälder" (Abschnitt 3.2) aus dem
        Skriptum \textit{Mathematik: Diskrete Strukturen} (Version v4.10 oder höher) und beantworten Sie
        folgende Fragen:
    \begin{enumerate}[label=(\alph*)]
        %Task (a)
        \item Ist jeder zusammenhängende Graph mit n Knoten und $n - 1$ Kanten eine Baum? Be-
        gründen Sie Ihre Antwort.\\
        Nach Voraussetzung jeder Knoten ist mit einander mit ein Kanten verbunden. Dann nach Definition 3.10 (MDS),
        da dieser Graph ist zusammenhangend und kreisfrei -- das ist ein Baum.

        %Task (b)
        \item Kein Antwort

        %Task (c)
        \item Welchen Prüfer-Code besitzt der folgende, markierte Baum?\\\\
        \begin{tikzpicture}
            [scale=.8,auto=left,every node/.style={circle,fill=blue!20}]
            \node (n2) at (1,1) {2};
            \node (n1) at (3,1) {1};
            \node (n7) at (5,1) {7};
            \node (n5) at (7,1) {5};
            \node (n4) at (3,-1) {4};
            \node (n6) at (2,-2) {6};
            \node (n8) at (4,-2) {8};
            \node (n3) at (5,-1) {3};
            \node (n9) at (6,-2) {9};

            \foreach \from/\to in {n2/n1, n1/n7, n7/n5, n1/n4, n4/n6, n4/n8, n7/n3, n3/n9}
            \draw (\from) -- (\to);
        \end{tikzpicture}\\\\
        Prüfer-Code: $17744173$

        %Task (d)
        \item Welcher markierte Baum hat den Prüfer-Code 212323212?\\\\
        \begin{tikzpicture}
            [scale=.8,auto=left,every node/.style={circle,fill=blue!20}]
            \node (n5) at (1,1) {5};
            \node (n10) at (3,1) {10};
            \node (n1) at (2,0) {1};
            
            \node (n11) at (1,-1) {11};
            \node (n7) at (5,-1) {7};
            
            \node (n4) at (1,-2) {4};
            \node (n2) at (2,-2) {2};
            \node (n3) at (4,-2) {3};
            
            \node (n6) at (1,-3) {6};
            \node (n8) at (2,-3) {8};
            \node (n9) at (5,-3) {9};

            \foreach \from/\to in {n5/n1, n10/n1, n1/n2, n2/n11, n2/n4, n2/n6, n2/n8, n2/n3, n3/n7, n9/n3}
            \draw (\from) -- (\to);
        \end{tikzpicture}\\\\
    \end{enumerate}
\end{document}






