% Math symbols and examples: http://en.wikibooks.org/wiki/LaTeX/Mathematics
% Also useful link: http://en.wikibooks.org/wiki/LaTeX/Advanced_Mathematics

% This document is used for title generating of exercise solutions. 
% Also it used for generating styles and environmentof documents.
% To add this title in document add next line:
% % This document is used for title generating of exercise solutions. 
% Also it used for generating styles and environmentof documents.
% To add this title in document add next line:
% % This document is used for title generating of exercise solutions. 
% Also it used for generating styles and environmentof documents.
% To add this title in document add next line:
% \input{../template.tex}
\documentclass[a4paper]{article}
\usepackage[utf8]{inputenc}
\usepackage[top=1.5cm]{geometry}
\usepackage{amssymb}
\usepackage{enumitem}
\usepackage{amsmath}

\allowdisplaybreaks

\DeclareMathOperator{\ggT}{ggT}
\DeclareMathOperator{\kgV}{kgV}

\title{Mathematik: Diskrete Strukturen \\ \Large Lösungsblatt}
\author{Anton Bubnov, Eugen Kuzmenko}

\documentclass[a4paper]{article}
\usepackage[utf8]{inputenc}
\usepackage[top=1.5cm]{geometry}
\usepackage{amssymb}
\usepackage{enumitem}
\usepackage{amsmath}

\allowdisplaybreaks

\DeclareMathOperator{\ggT}{ggT}
\DeclareMathOperator{\kgV}{kgV}

\title{Mathematik: Diskrete Strukturen \\ \Large Lösungsblatt}
\author{Anton Bubnov, Eugen Kuzmenko}

\documentclass[a4paper]{article}
\usepackage[utf8]{inputenc}
\usepackage[top=1.5cm]{geometry}
\usepackage{amssymb}
\usepackage{enumitem}
\usepackage{amsmath}

\allowdisplaybreaks

\DeclareMathOperator{\ggT}{ggT}
\DeclareMathOperator{\kgV}{kgV}

\title{Mathematik: Diskrete Strukturen \\ \Large Lösungsblatt}
\author{Anton Bubnov, Eugen Kuzmenko}


\begin{document}
    \maketitle
    \section*{Vertiefung:}
    \begin{enumerate}[label=(\alph*)]
        % Task (a)
        \item Welche Potenzreihe (in möglichst einfacher Darstellung mit nur einem Summensymbol) 
        entspricht dem Produkt der Potenzreihen $\displaystyle\sum_{n=0}^{\infty} x^n$ und 
        $\displaystyle\sum_{n=0}^{\infty} (2n+1)\cdot x^n$?\\
        \begin{align*}
            \sum_{n=0}^{\infty} x^n \cdot \sum_{n=0}^{\infty} (2n+1)\cdot x^n &= 
            \sum_{n=0}^{\infty} \Bigg(\sum_{k=0}^{n} (2n+1)\Bigg) \cdot x^n &\textrm{(Potenzreihen Rechnenregeln)}\\
            &= \sum_{n=0}^{\infty} (n+1)^2 \cdot x^n &\textrm{(BK Proposition 3.B)}
        \end{align*}

        % Task (b)
        \item Keine Antwort

        % Task (c)
        \item Welche erzeugende Funktion hat die Folge $(1, 1, 1, 1, 1, 1, 0, 0, \ldots , 0, \ldots)$?
        \begin{align*}
            \sum_{n=0}^{5} x^n
            &=\sum_{n=0}^{\infty} x^n - \sum_{n=6}^{\infty} x^n \\
            &=\sum_{n=0}^{\infty} x^n - \sum_{n=0}^{\infty} x^{n+6} \\
            &=\sum_{n=0}^{\infty} x^n - x^6 \cdot \sum_{n=0}^{\infty} x^n \\
            &=\frac{1}{1-x} - \frac{x^6}{1-x} \\
            &=\frac{1-x^6}{1-x}
        \end{align*}

        % Task (d)
        \item Welche erzeugende Funktion hat die Folge $(0, 0, 0, 0, 0, 0, 1, 1, \ldots , 1,\ldots)$?
        \begin{align*}
            \sum_{n=5}^{\infty} x^n
            &=\sum_{n=0}^{\infty} x^{n+5} \\
            &=x^5\cdot\sum_{n=0}^{\infty} x^n \\
            &=\frac{x^5}{1-x}
        \end{align*}

        % Task (e)
        \item Keine Antwort

        % Task (f)
        \item Welche erzeugende Funktion hat die Folge $(-1, 1, -1, 1, -1, 1,\ldots, -1, 1,\ldots)$?
        \begin{align*}
            &a_n = (-1)^{n+1} \\
            A(x)&=\sum_{n=0}^{\infty} (-1)^{n+1} x^n \\
            &=(-1)\sum_{n=0}^{\infty} (-1)^n x^n \\
            &=(-1)\cdot \frac{1}{1+x}
        \end{align*}

        % Task (g)
        \item Welche erzeugende Funktion hat die Folge $(0,1,3,6,10,15,21,28,36,\ldots,?,\ldots)$?
        \begin{align*}
            &a_n = \Big(\frac{1}{2}\Big)(n+1)n  \\
            A(x)&=\sum_{n=0}^{\infty} \frac{n(n+1)}{2} x^n \\
            &=\frac{1}{2}\cdot\sum_{n=0}^{\infty} (n^2+n) x^n \\
            &=\frac{1}{2}\cdot\Big(\sum_{n=0}^{\infty} n^2 x^n  + \sum_{n=0}^{\infty} n x^n\Big)\\
            &=\frac{x(1+x) + 1 -x }{2(1-x)^3} \\
            &=\frac{x^2 + 1}{2(1-x)^3}
        \end{align*}

        % Task (h)
        \item Ist $a_n =_{def} (1+\sqrt 2)^n+(1-\sqrt 2)^n$ für alle $n\in \mathbb{N}$ stets eine natürliche Zahl?\\\\
        Verwenden wir ein Binomische Formel:
        \begin{align*}
            a_n &= (1+\sqrt 2)^n+(1-\sqrt 2)^n \\
            &=\sum_{k=0}^{n} \binom{n}{k} 1^{n-k}\cdot\sqrt{2}^k +
            \sum_{k=0}^{n} \binom{n}{k} 1^{n-k}\cdot(-\sqrt{2})^k \\
            &= \sum_{k=0}^{n} \binom{n}{k}\sqrt{2}^k +
            \sum_{k=0}^{n} \binom{n}{k}(-\sqrt{2})^k \\
            &= \sum_{k=0}^{n} \binom{n}{k}\cdot\big(\sqrt{2}^k + (-\sqrt{2})^k\big)
        \end{align*}
        Da Binomialkoeffizient immer eine natürliche Zahle ist; Quadratwurzel mit gerade $k$ aus 2 
        gibt uns die Summe von zwei natürlichen Zahlen, Quadratwurzel mit ungerade $k$ abkürzt einfach
        die Quadratwurzeln. Folglich $a_n \in \mathbb{N}$.

        % Task (i)
        \item Gegeben sei folgende lineare Rekursionsgleichung (aber ohne fester Ordnung!):
        \begin{align*}
            a_n &=_{def} 1 + \sum_{k=0}^{n-1} a_k \\
            a_0 &=_{def} 1
        \end{align*}
        Bestimmen Sie die Folgenglieder $a_n$ mittels der erzeugenden Funktion.\\\\
        Ausschreiben wir erste 5 Folgenglieder:
        \begin{align*}
            a_0 &= 1\\
            a_1 &= 2\\
            a_2 &= 4\\
            a_3 &= 8\\
            a_4 &= 16
        \end{align*}
        Da kann man deutlich sehen, dass $a_n = 2^n$ gilt. Sofern wir die Folge $(1,2,4,6,8,16,\ldots)$
        haben erzeugende Funktion für solche Folge ist $\displaystyle\frac{1}{1-2z}$.

        % Task (j)
        \item Bestimmen Sie die Folgenglieder zur Folge aus Teilaufgabe (i), indem Sie $a_n - a_n-1$
        betrachten und eine einfacherere Rekursionsgleichung gewinnen.
        \begin{align*}
            a_n - a_{n-1} &= 1+\sum_{k=0}^{n-1} a_k - \Big(1+\sum_{k=0}^{n-2} a_k\Big) \\
            &= \sum_{k=0}^{n-1} a_k - \sum_{k=0}^{n-2} a_k \\
            &= a_{n-1} \\
            \implies a_n - a_{n-1} &= a_{n-1} \\
            a_n &= 2\cdot a_{n-1}
        \end{align*}
    \end{enumerate}
    \section*{Kreativität:}
    \begin{enumerate}[label=(\alph*)]
        %Task (a)
        \item Task a
    \end{enumerate}
    \section*{Transfer:}
    \begin{enumerate}[label=(\alph*)]
        \item Task a
    \end{enumerate}
\end{document}






