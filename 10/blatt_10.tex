% Math symbols and examples: http://en.wikibooks.org/wiki/LaTeX/Mathematics
% Also useful link: http://en.wikibooks.org/wiki/LaTeX/Advanced_Mathematics

% This document is used for title generating of exercise solutions. 
% Also it used for generating styles and environmentof documents.
% To add this title in document add next line:
% % This document is used for title generating of exercise solutions. 
% Also it used for generating styles and environmentof documents.
% To add this title in document add next line:
% % This document is used for title generating of exercise solutions. 
% Also it used for generating styles and environmentof documents.
% To add this title in document add next line:
% \input{../template.tex}
\documentclass[a4paper]{article}
\usepackage[utf8]{inputenc}
\usepackage[top=1.5cm]{geometry}
\usepackage{amssymb}
\usepackage{enumitem}
\usepackage{amsmath}

\allowdisplaybreaks

\DeclareMathOperator{\ggT}{ggT}
\DeclareMathOperator{\kgV}{kgV}

\title{Mathematik: Diskrete Strukturen \\ \Large Lösungsblatt}
\author{Anton Bubnov, Eugen Kuzmenko}

\documentclass[a4paper]{article}
\usepackage[utf8]{inputenc}
\usepackage[top=1.5cm]{geometry}
\usepackage{amssymb}
\usepackage{enumitem}
\usepackage{amsmath}

\allowdisplaybreaks

\DeclareMathOperator{\ggT}{ggT}
\DeclareMathOperator{\kgV}{kgV}

\title{Mathematik: Diskrete Strukturen \\ \Large Lösungsblatt}
\author{Anton Bubnov, Eugen Kuzmenko}

\documentclass[a4paper]{article}
\usepackage[utf8]{inputenc}
\usepackage[top=1.5cm]{geometry}
\usepackage{amssymb}
\usepackage{enumitem}
\usepackage{amsmath}

\allowdisplaybreaks

\DeclareMathOperator{\ggT}{ggT}
\DeclareMathOperator{\kgV}{kgV}

\title{Mathematik: Diskrete Strukturen \\ \Large Lösungsblatt}
\author{Anton Bubnov, Eugen Kuzmenko}

\usepackage{ tipa }

\begin{document}
    \maketitle
    \section*{Vertiefung:}
    \begin{enumerate}[label=(\alph*)]
        % Task (a)
        \item Bestimmen Sie \textchi$(Q_3 )$.\\
        Da $deg_G(V) = 3 $ braucht man mindestens 3 Farben. Sofern $(Q_3)$ aus zwei gebundene Quadranten besteht brauchen wir maximal 3 Farben, also \textchi$(Q_3) = 3$, weil für jeder Quadrant wir zwei Farben brauchen und noch eine für die Verbindung diese Quadranten.
        % ich meine mal 2 Farben reichen :) vergleiche dazu Blatt 9 (j)
        % $\chi (Q_3) = 2$ 
                
        % Task (b)
        \item Bestimmen Sie $ \chi (Q_4) $
                
        $\chi (Q_4) = 2 $
                
        % Task (c)
        \item
               
        % Task (d)
        \item 
        Bestimmen Sie $\chi'(Q_3) $
        Nach Vizing folgt: $\chi'(Q_3) = 3 $(Theorem 3.31)
        
    \end{enumerate}
    \section*{Kreativität:}
    \begin{enumerate}[label=(\alph*)]
    	%Task (a)
    	\item Task a
    \end{enumerate}
    \section*{Transfer:}
    \begin{enumerate}[label=(\alph*)]
    	\item Task a
    \end{enumerate}
\end{document}






