% Math symbols and examples: http://en.wikibooks.org/wiki/LaTeX/Mathematics
% Also useful link: http://en.wikibooks.org/wiki/LaTeX/Advanced_Mathematics

% This document is used for title generating of exercise solutions. 
% Also it used for generating styles and environmentof documents.
% To add this title in document add next line:
% % This document is used for title generating of exercise solutions. 
% Also it used for generating styles and environmentof documents.
% To add this title in document add next line:
% % This document is used for title generating of exercise solutions. 
% Also it used for generating styles and environmentof documents.
% To add this title in document add next line:
% \input{../template.tex}
\documentclass[a4paper]{article}
\usepackage[utf8]{inputenc}
\usepackage[top=1.5cm]{geometry}
\usepackage{amssymb}
\usepackage{enumitem}
\usepackage{amsmath}

\allowdisplaybreaks

\DeclareMathOperator{\ggT}{ggT}
\DeclareMathOperator{\kgV}{kgV}

\title{Mathematik: Diskrete Strukturen \\ \Large Lösungsblatt}
\author{Anton Bubnov, Eugen Kuzmenko}

\documentclass[a4paper]{article}
\usepackage[utf8]{inputenc}
\usepackage[top=1.5cm]{geometry}
\usepackage{amssymb}
\usepackage{enumitem}
\usepackage{amsmath}

\allowdisplaybreaks

\DeclareMathOperator{\ggT}{ggT}
\DeclareMathOperator{\kgV}{kgV}

\title{Mathematik: Diskrete Strukturen \\ \Large Lösungsblatt}
\author{Anton Bubnov, Eugen Kuzmenko}

\documentclass[a4paper]{article}
\usepackage[utf8]{inputenc}
\usepackage[top=1.5cm]{geometry}
\usepackage{amssymb}
\usepackage{enumitem}
\usepackage{amsmath}

\allowdisplaybreaks

\DeclareMathOperator{\ggT}{ggT}
\DeclareMathOperator{\kgV}{kgV}

\title{Mathematik: Diskrete Strukturen \\ \Large Lösungsblatt}
\author{Anton Bubnov, Eugen Kuzmenko}


\begin{document}
    \maketitle
    \section*{Vertiefung:}
    \begin{enumerate}[label=(\alph*)]
        % Task (a)
        \item Drücken Sie die Anzahl der surjektiven Funktionen 
        $f : \{0, 1\}^n \to \{0, 1\}^2$ mit Hilfe der Stirling-Zahlen zweiter Art aus.

        % Task (b)
        \item Kein Antwort

        % Task (c)
        \item Kein Antwort

        % Task (d)
        \item Für drei Mengen $A, B$ und $C$ gelten folgende Eigenschaften: 
        $||A|| = 63, ||B|| = 91, ||C|| = 44, ||A\cap B|| = 25, ||A\cap C|| = 23, ||C \cap B|| = 21$. 
        Außerdem gelte $||A\cup B\cup C|| = 139$. Wie groß ist $||A \cap B \cap C||$?\\\\
        Nach Theorem 1.19 Beispiel es gilt:
        \begin{align*}
        ||A\cup B\cup C|| &= ||A|| + ||B|| + ||C|| - ||A\cap B|| - ||A\cap C|| - ||C \cap B|| + ||A \cap B \cap C||\\
        \Rightarrow ||A \cap B \cap C|| &= ||A\cup B\cup C|| - ||A|| - ||B|| - ||C|| + ||A\cap B|| + ||A\cap C|| + ||C \cap B|| \\
        \Rightarrow ||A \cap B \cap C|| &= 139 - 63 - 91 - 44 + 25 + 23 + 21 = 10 \\
        \end{align*}
    \end{enumerate}
\end{document}