% Math symbols and examples: http://en.wikibooks.org/wiki/LaTeX/Mathematics
% Also useful link: http://en.wikibooks.org/wiki/LaTeX/Advanced_Mathematics

% This document is used for title generating of exercise solutions. 
% Also it used for generating styles and environmentof documents.
% To add this title in document add next line:
% % This document is used for title generating of exercise solutions. 
% Also it used for generating styles and environmentof documents.
% To add this title in document add next line:
% % This document is used for title generating of exercise solutions. 
% Also it used for generating styles and environmentof documents.
% To add this title in document add next line:
% \input{../template.tex}
\documentclass[a4paper]{article}
\usepackage[utf8]{inputenc}
\usepackage[top=1.5cm]{geometry}
\usepackage{amssymb}
\usepackage{enumitem}
\usepackage{amsmath}

\allowdisplaybreaks

\DeclareMathOperator{\ggT}{ggT}
\DeclareMathOperator{\kgV}{kgV}

\title{Mathematik: Diskrete Strukturen \\ \Large Lösungsblatt}
\author{Anton Bubnov, Eugen Kuzmenko}

\documentclass[a4paper]{article}
\usepackage[utf8]{inputenc}
\usepackage[top=1.5cm]{geometry}
\usepackage{amssymb}
\usepackage{enumitem}
\usepackage{amsmath}

\allowdisplaybreaks

\DeclareMathOperator{\ggT}{ggT}
\DeclareMathOperator{\kgV}{kgV}

\title{Mathematik: Diskrete Strukturen \\ \Large Lösungsblatt}
\author{Anton Bubnov, Eugen Kuzmenko}

\documentclass[a4paper]{article}
\usepackage[utf8]{inputenc}
\usepackage[top=1.5cm]{geometry}
\usepackage{amssymb}
\usepackage{enumitem}
\usepackage{amsmath}

\allowdisplaybreaks

\DeclareMathOperator{\ggT}{ggT}
\DeclareMathOperator{\kgV}{kgV}

\title{Mathematik: Diskrete Strukturen \\ \Large Lösungsblatt}
\author{Anton Bubnov, Eugen Kuzmenko}


\DeclareMathOperator{\ggT}{ggT}
\DeclareMathOperator{\kgV}{kgV}

\begin{document}
    \maketitle
    \section*{Vertiefung:}
    \begin{enumerate}[label=(\alph*)]
        % Task (a)
        \item Bei einem Turnier spielen n Mannschaften "jeder gegen jeden" (genau einmal). Für einen
		Sieg gibt es 3 Punkte, für ein Unentschieden 1 Punkt, für eine Niederlage 0 Punkte. Die
		Platzierungen ergeben sich aus den erzielten Punkten, bei Punktgleichheit wird gelost.
		Kann eine Mannschaft mit n Punkten Turnierletzter werden? \\\\
		Es kann eine Situation sein, wenn alle Mannschaften gleiche Anzahl der Punkten haben. 
		Ein Mannschaft spielt insgesamt $n-1$ Spielen. Um $n$ Punkten zu bekommen kann die einmal 
		gewinnen, und einmal verloren (3 Punkten zusammen). Dann hat die noch $n-3$ Spielen, die die 
		Unentschieden spielen kann ($n-3$ Punkten zusammen). Damit in die Summe haben wir genau $n$ 
		Punkten. Es kann sein, dass alle Mannschaften so spielen werden, weil diese Situatuion
		symmetrisch ist. Dann nach Voraussetzung, wenn alle Mannschaften gleiche Anzahl der Punkten 
		haben, wird es gelost. In diesem Fall kann die Mannschaft Turnierletzer sein.
		
		% Task (b)
        \item Keine Antwort
        
        % Task (c)
        \item Wie viele Punkte kann der Turnierzweite höchstens erreichen, wenn das gleiche Szenario
		wie in Teilaufgabe (a) betrachtet wird? \\\\
		Turniererster kann maximal $(n-1)\cdot 3$ Punkten bekommen, wenn der in alle Spiele gewinnen 
		wird. Dann Turnierzweite muss einmal verlieren. D.h. dass Turnierzweite kann maximal 
		$(n-2)\cdot 3$ bekommen.
        
        % Task (d)
        \item Wie viele Mannschaften können höchstens $n + 1$ Punkte erreichen, wenn das gleiche
		Szenario wie in Teilaufgabe (a) betrachtet wird? \\\\
		  
        
        % Task (e)
        \item Keine Antwort
        
        % Task (f)
        \item Keine Antwort
        
        % Task (g)
        \item Keine Antwort
        
        % Task (h)
        \item Wie viele rekursive Aufrufe benötigt der Algorithmus von Euklid, um $\ggT(F_{k+2},F_{k+4})$
		zu bestimmen. Hierbei steht $F_k$ für die $k$-te Fibonacci-Zahl.
		Es gilt:
		\[\bmod(F_{k+4},F_{k+2}) = \bmod(F_{k+2} + F_{k+3},F_{k+2}) = \bmod(F_{k+3},F_{k+2}) = F_{k+1}\]
		Dann folgt:
		\begin{align*}
			\ggT(F_{k+2},F_{k+4}) &= EUKLID(F_{k+2},F_{k+4}) \\
			 &= EUKLID(\bmod(F_{k+4},F_{k+2}),F_{k+2}) &(1) \\ 
			 &= EUKLID(F_{k+1},F_{k+2}) &(2)
		\end{align*}
		Da die Schritt (1) ein Aufruf braucht und Schritt (2) braucht $k-1$ Aufrufe, haben wir insgesamt
		$k$ rekursive Aufrufe.

    \end{enumerate}
    \section*{Kreativität:}
    \begin{enumerate}[label=(\alph*)]
    	%Task (a)
    	\item Task a
    \end{enumerate}
    \section*{Selbststudium:}
    \begin{enumerate}[label=(\alph*)]
    	\item $x_n = 3x_{n-1}+ \sqrt{2}$, fuer $n \geq 1 $ ; $ x_0 = 1.$ \\ \\
    	Ihomogen, erster Ordnung \\
    	$ x_o = b_0 =1 ; \quad a=3; \quad b_1 = \sqrt{2} $ 
    	\[x_n =1 \cdot 3^n + \sqrt{2} \cdot \frac{3^n -1}{2} \]
    	\item $x_n = 2x_{n-2}$, fuer $n \geq 2 $ ;  $\quad x_1 = 2, \quad x_0 =1 .$ \\ \\
    	Homogen, zweiter Ordnung.\\
    	$x_1 =b_1 = 2, \quad x_0=b_0=1, \quad a_1=0, \quad a_2=2$
    	\begin{align*}
    		&t^2-0 \cdot t -2 =0 \\
    		&t= \pm \sqrt{2} ; \quad \alpha = \sqrt{2}, \quad \beta = -\sqrt{2} \\
    		&A = \frac{2-1\cdot (- \sqrt{2} )}{\sqrt{2}+\sqrt{2}} \quad B = \frac{2-1\cdot  \sqrt{2} }{\sqrt{2}+\sqrt{2}} \\\\
    		&x_n = \frac{2+\sqrt{2}}{2\sqrt{2}} \cdot (\sqrt{2})^n - \frac{2-\sqrt{2}}{2\sqrt{2}} \cdot (-\sqrt{2})^n
    	\end{align*}
    	\item $x_n = 2x_{n-1} +x_{n-2}$, fuer $n \geq 2 $ ;  $\quad x_1 = 2, \quad x_0 =1 .$ \\ \\
    	Homogen, zweiter Ordnung. \\
    	    	$x_1 =b_1 = 2, \quad x_0=b_0=1, \quad a_1=2, \quad a_2=1$\\
    	    	\begin{align*}
    	    		   &t^2-2 \cdot t -1 =0 \\
						&\alpha = 1+\sqrt{2}, \quad \beta = 1-\sqrt{2} \\
						&A= \frac{2-1+\sqrt{2}}{1+\sqrt{2-1+\sqrt{2}}} \quad B = \frac{2-1-\sqrt{2}}{2\sqrt{2}} \\ \\
						&x_n=\frac{1+\sqrt{2}}{2\sqrt{2}} \cdot(1+ \sqrt{2} )^n - \frac{1-\sqrt{2}}{2\sqrt{2}}(1-\sqrt{2})^n
    	    	\end{align*}
		\item $x_n = -3x_{n-1} +4x_{n-2}$, fuer $n \geq 2 $ ; mit $\quad x_1 = 1, \quad x_0 =0 .$ \\ \\
		Homogen, 2.Ordnung.
	    $x_1 =b_1 = 1, \quad x_0=b_0=0, \quad a_1=-3, \quad a_2=4$\\
	  	\begin{align*}
			   &t^2+3 \cdot t -4 =0 \\
			   &d=b^2-4ac= 9 \cdot -4 \cdot (-4) = 25 \\
			   &t_{1,2}= \frac{-3 \pm \sqrt{25}}{2}; \quad \alpha = -4, \quad \beta =1 \\
			   &A= \frac{1-0}{-4-1} = -\frac{1}{5}; \quad B=-\frac{1}{5}\\
			   &x_n=-\frac{1}{5}(-4)^n - (-\frac{1}{5})1=(- \frac{1}{5})((-4)^n-1)=\frac{1}{5}(1-(-4)^n)
		\end{align*}
 \item $x_n = -3x_{n-1} +4x_{n-2}$, fuer $n \geq 2 $ ; mit $\quad x_1 = 0, \quad x_0 =1 .$ \\ \\
   Homogen, 2.Ordnung.
	    $x_1 =b_1 = 0, \quad x_0=b_0=1, \quad a_1=-3, \quad a_2=4$\\
	    \begin{align*}
	   	   &t^2+3 \cdot t -4 =0 \\
	   	   &\alpha = -4, \quad \beta =1 \\
	   	   &A= \frac{0-1}{-5} = -\frac{1}{5}; \quad B=-\frac{4}{5}\\
	   	   &x_n=\frac{1}{5}(-4)^n - (-\frac{4}{5})1^n=\frac{1}{5}(-4)^n+\frac{4}{5}
	   	   	\end{align*}
    \end{enumerate}
\end{document}






