% Math symbols and examples: http://en.wikibooks.org/wiki/LaTeX/Mathematics
% Also useful link: http://en.wikibooks.org/wiki/LaTeX/Advanced_Mathematics

% This document is used for title generating of exercise solutions. 
% Also it used for generating styles and environmentof documents.
% To add this title in document add next line:
% % This document is used for title generating of exercise solutions. 
% Also it used for generating styles and environmentof documents.
% To add this title in document add next line:
% % This document is used for title generating of exercise solutions. 
% Also it used for generating styles and environmentof documents.
% To add this title in document add next line:
% \input{../template.tex}
\documentclass[a4paper]{article}
\usepackage[utf8]{inputenc}
\usepackage[top=1.5cm]{geometry}
\usepackage{amssymb}
\usepackage{enumitem}
\usepackage{amsmath}

\allowdisplaybreaks

\DeclareMathOperator{\ggT}{ggT}
\DeclareMathOperator{\kgV}{kgV}

\title{Mathematik: Diskrete Strukturen \\ \Large Lösungsblatt}
\author{Anton Bubnov, Eugen Kuzmenko}

\documentclass[a4paper]{article}
\usepackage[utf8]{inputenc}
\usepackage[top=1.5cm]{geometry}
\usepackage{amssymb}
\usepackage{enumitem}
\usepackage{amsmath}

\allowdisplaybreaks

\DeclareMathOperator{\ggT}{ggT}
\DeclareMathOperator{\kgV}{kgV}

\title{Mathematik: Diskrete Strukturen \\ \Large Lösungsblatt}
\author{Anton Bubnov, Eugen Kuzmenko}

\documentclass[a4paper]{article}
\usepackage[utf8]{inputenc}
\usepackage[top=1.5cm]{geometry}
\usepackage{amssymb}
\usepackage{enumitem}
\usepackage{amsmath}

\allowdisplaybreaks

\DeclareMathOperator{\ggT}{ggT}
\DeclareMathOperator{\kgV}{kgV}

\title{Mathematik: Diskrete Strukturen \\ \Large Lösungsblatt}
\author{Anton Bubnov, Eugen Kuzmenko}


\begin{document}
    \maketitle
    \section*{Vertiefung:}
    \begin{enumerate}[label=(\alph*)]
        % Task (a)
        \item Welchen Wert hat $\pi^{23}(5)$ f\"ur $\pi=(4 2 5 3 1)(8 6 7)$ ?\\\\
        Da 5 befindet sich in dem Zyklus der Länge 5, gilt $\pi^{23}(5)=\pi^{3}(5)$
        \\
               \begin{align*}
        \pi^{0}(5) &= 5\\
        \pi^1(5) &= 1 \\
        \pi^2(5) &= 4\\
        \pi^3(5) &= 3\\
        \textrm {BIN NICHT  SICHER OB die ANTWORT 3 ODER 5 } \pi^4(5) &= 5\\
        \pi^{23}(5)&=3.
               \end{align*}
        % Task (b)
        \item Wie sieht die Permutation (3, 2, 6, 7, 5, 1, 4) in Zyklenschreibweise aus?\\\\
              \begin{align*}
              \pi(1) &= 3 \quad \pi(3)=6 \quad \pi(6)=1 \\
              \pi(2) &= 2\\
              \pi(4) &= 7 \quad \pi(7)=4 \\
              \pi(5) &= 5  
              \end{align*}
        $(361)(2)(74)(5)$
        % Task (c)
        \item Wie sieht die Permutation (8 1 5 3)(2 4)(6 7) in Tupelschreibweise aus?\\\\
        \begin{align*}
        	\pi(8) &=1 \quad \pi(1)= 5 \quad \pi(5)=3 \quad \pi(3)=8\\
        	\pi(2) &=4 \quad \pi(4)=2\\
        	\pi(7) &=7 \quad \pi(7)=6
        \end{align*}
        $(5, 4,8,2,3,7,6,1)$
        % Task (d)
        \item Welchen Zyklentyp besitzt die Permutation (2, 4, 1, 3, 5, 8, 6, 9, 7)?\\\\
        Es gibt folgende Zyklen: (2 4 3 1)(5)(8 9 7 6), der Zyklentyp ist (4, 4, 1), oder $4^2 1^1$.
        % Task (e)
        \item Wie viele Permutationen von $n$ Elementen mit dem Zyklentyp 
        $(3, 1, \ldots, 1)$ gibt es für $n \geq 4$?\\\\
        Wir sollen erstens ein Tupel mit 3 Elementen aus $n$ Elementen machen: 
        \[\binom{n}{3} = \frac{n!}{3!(n-3)!} 
        \textrm{\qquad(nach Satz 6. (ungeordnet, ohne Zur\"ucklegen) und Bin. Def.)}\]
        In dieser Tupel aus 3 Elementen haben wir 2 m\"ogliche Permutationen 
        (nach Satz 13., Zyklenschreibweise (Beispiel 3)). Folglich:
        \[\frac{n!}{3!(n-3)!} \cdot 2 = \frac{n!}{3\cdot(n-3)!}\]
    \end{enumerate}
\end{document}