% Math symbols and examples: http://en.wikibooks.org/wiki/LaTeX/Mathematics
% Also useful link: http://en.wikibooks.org/wiki/LaTeX/Advanced_Mathematics

% This document is used for title generating of exercise solutions. 
% Also it used for generating styles and environmentof documents.
% To add this title in document add next line:
% % This document is used for title generating of exercise solutions. 
% Also it used for generating styles and environmentof documents.
% To add this title in document add next line:
% % This document is used for title generating of exercise solutions. 
% Also it used for generating styles and environmentof documents.
% To add this title in document add next line:
% \input{../template.tex}
\documentclass[a4paper]{article}
\usepackage[utf8]{inputenc}
\usepackage[top=1.5cm]{geometry}
\usepackage{amssymb}
\usepackage{enumitem}
\usepackage{amsmath}
\usepackage{fancyhdr}


\allowdisplaybreaks

\DeclareMathOperator{\ggT}{ggT}
\DeclareMathOperator{\kgV}{kgV}

\pagestyle{fancy}
\fancyhf{}
\rhead{Anton Bubnov, Yevgen Kuzmenko}
\lhead{Mathematik: Diskrete Strukturen}
\cfoot{\thepage}

\title{Mathematik: Diskrete Strukturen \\ \Large Lösungsblatt}
\author{Anton Bubnov, Yevgen Kuzmenko}

\documentclass[a4paper]{article}
\usepackage[utf8]{inputenc}
\usepackage[top=1.5cm]{geometry}
\usepackage{amssymb}
\usepackage{enumitem}
\usepackage{amsmath}
\usepackage{fancyhdr}


\allowdisplaybreaks

\DeclareMathOperator{\ggT}{ggT}
\DeclareMathOperator{\kgV}{kgV}

\pagestyle{fancy}
\fancyhf{}
\rhead{Anton Bubnov, Yevgen Kuzmenko}
\lhead{Mathematik: Diskrete Strukturen}
\cfoot{\thepage}

\title{Mathematik: Diskrete Strukturen \\ \Large Lösungsblatt}
\author{Anton Bubnov, Yevgen Kuzmenko}

\documentclass[a4paper]{article}
\usepackage[utf8]{inputenc}
\usepackage[top=1.5cm]{geometry}
\usepackage{amssymb}
\usepackage{enumitem}
\usepackage{amsmath}
\usepackage{fancyhdr}


\allowdisplaybreaks

\DeclareMathOperator{\ggT}{ggT}
\DeclareMathOperator{\kgV}{kgV}

\pagestyle{fancy}
\fancyhf{}
\rhead{Anton Bubnov, Yevgen Kuzmenko}
\lhead{Mathematik: Diskrete Strukturen}
\cfoot{\thepage}

\title{Mathematik: Diskrete Strukturen \\ \Large Lösungsblatt}
\author{Anton Bubnov, Yevgen Kuzmenko}

% Used to draw graphs. More info: http://en.wikipedia.org/wiki/PGF/TikZ
% Examples: http://www.texample.net/tikz/examples/
\usepackage{tikz} 
\newcommand{\floor}[1]{\left\lfloor #1\right\rfloor}
\newcommand{\ceil}[1]{\left\lceil #1 \right\rceil}

\begin{document}
    \maketitle
    \section*{Vertiefung:}
    \begin{enumerate}[label=(\alph*)]
        % Task (a) 
        \item Welche Gradfolge besitzt $M_{3,4}$?\\
        Die Gradfolge von $M_{3,4}$ ist $4,4,3,3,3,3,3,3,2,2,2,2$.

        % Task (b)
        \item Gibt es einen Graphen mit Gradfolge $(5, 4, 3, 2, 2, 2, 2)$? - Und wenn ja, welchen?\\
        Ja es gibt einen solchen Graphen. Folgender Graph würde die Gradfolge erfüllen: \\
        \begin{tikzpicture}
            [scale=.8,auto=left,every node/.style={circle,fill=blue!20}]
            \node (n1) at (3,1) {5};
            \node (n2) at (1,-1) {2};
            \node (n3) at (3,-1) {2};
            \node (n4) at (4,-1) {2};
            \node (n6) at (5,-1) {2};
            \node (n7) at (2,-2) {4};
            \node (n8) at (4,-2) {3};

            \foreach \from/\to in {n1/n2, n1/n3, n1/n4, n1/n6, n7/n1, n7/n2, n7/n3, n7/n8, n8/n6, n8/n4}
            \draw (\from) -- (\to);
        \end{tikzpicture}

        % Task (c)
        \item Gibt es einen Graphen mit Gradfolge $(5, 4, 3, 2, 2, 2, 2, 1)$? - Und wenn ja, welchen?\\
        Nach Proposition 3.3 gilt, dass die Summe der Grade über die Menge der Knoten eine gerade Zahl ergibt. 
        In unserem Fall beträgt die Summe der Grade 21, was eine ungerade Zahl ist. Folglich kann es 
        keinen solchen Graphen geben. 

        % Task (d)
        \item Gibt es einen Graphen mit Gradfolge $(5, 2, 1, 1, 1, 1, 1, 1, 1)$? - Und wenn ja, welchen?\\
        Es kann keinen solchen zusammenh"angenden Graphen geben (wenn es nicht zusammenh"angend sein muss, 
        dann siehe Beispiel). Der Graph hätte 9 Knoten und (nach Proposition 3.3) 7 Kanten. 
        Folgende Graphik zeigt ein Beispiel für unseren Graphen: \\\\
        \begin{tikzpicture}
            [scale=.8,auto=left,every node/.style={circle,fill=blue!20}]
            \node (n1) at (3,1) {5};
            \node (n2) at (1,-1) {1};
            \node (n3) at (2,-1) {1};
            \node (n4) at (3,-1) {1};
            \node (n5) at (4,-1) {1};
            \node (n6) at (5,-1) {1};
            \node (n7) at (6,-1) {1};
            \node (n8) at (7,-1) {1};

            \node (n9) at (6,1) {2};

            \foreach \from/\to in {n1/n2, n1/n3, n1/n4, n1/n5, n1/n6, n9/n8, n9/n7}
            \draw (\from) -- (\to);
        \end{tikzpicture}\\\\
        Es wird offensichtlich, dass es keine Möglichkeit zum Verbinden existiert. Ohne den Knoten mit 
        den 5 Verbindungen bleiben genau 2 Kanten übrig. Mit zwei Kanten lässt sich nur ein zusammenhängender Graph mit höchstens 3 Knoten erzeugen. DIe 2 Kanten reichen nicht aus, um die 
        (verbliebenen) 8 Knoten zu verbinden. Ein Graph mit der Gradfolge $(5,2,1,1,1,1,1,1,1)$ kann es 
        folglich nicht geben. 
         
        % Task (e)
        \item Wie groß ist der maximale Abstand zweier Knoten im Hyperwürfel Q$_d$? \\
        Der maximale Abstand zweier Knoten im Hyperwürfel Q$_d$ ist $d$.
        
        % Task (f)
        \item Wie groß ist der maximale Abstand zweier Knoten im Gittergraphen $M_{n,n}$ ? \\
        Der maximale Abstand zweier Knoten im Gittergraphen $M_{n,n}$ ist $2\cdot n - 2$
        
        % Task (g)
        \item Welche Knoten haben im $M_{n,n}$ den kleinsten maximalen Abstand zu einem anderen Knoten? \\
        Das sind die Knoten die in der "Mitte" des Graphen sind. Für ungerade $n$ ist das der Knoten mit der
        $(\big\lceil\frac{n}{2}\big\rceil,\big\lceil\frac{n}{2}\big\rceil)$ Koordinate. 
        Für gerade $n$ sind das die Knoten, die zu $M_{2,2}$ als Teilgraph gehören, wobei dieser Teilgraph 
        sich auch in die Mitte des Gittergraphen befindet.
        
        % Task (h)
        \item Welche Knoten haben im $M_{n,n}$ den größten maximalen Abstand zu einem anderen Knoten? \\
        Das sind die Knoten, die ''an den Ecken'' vom Graph sind. \\
        Folgende Knotenzählung (Zählung beginnt von 1 weg) ermittelt die "gesuchten Knoten":
        $$v_1, v_n, v_{n^2-n+1} \text{ und } v_{n^2}.$$
        
        % Task (i)
        \item Wie viele Wege der Länge $k$ enthält ein $r$-regulärer Graph mit $n$ Knoten? \\
        Aus der Vorraussetzung folgt, dass jeder Knoten r Nachbarn hat. Von jedem Knoten kann man also r verschieden Knoten für den nächsten Schritt wählen. Dies wird $k$ mal wiederholt, wodurch wir $r^k$ mögliche Wege erhalten. Da man von n verschiedenen Knoten starten kann, gibt es $n \cdot r^k$ mögliche Wege insgesamt.
        
        % Task (j)
        \item Wie viele Kreise der Länge $k$ enthält der vollständige Graph $K_n$? \\
        Kein Antwort
    \end{enumerate}

    \section*{Kreativität:}
Sei $V$ beliebig aber fest und gelte $|V| \ge 4$. Sei $H(V_H,E_H)$ ein beliebiger aber fester induzierter Teilgraph auf $G$ mit $|V_H|=4$. Sei $A=V_H \cap U$ und $B=V_H \cap (V\setminus U)$. Die Knoten aus diesen Mengen werden mit $a_i$ und $b_j$ bezeichnet. Wir wollen folgende Eigenschaften zeigen:
\begin{enumerate}[label=(\alph*)]
	\item Kein Teilgraph $H$ ist ein Kreis der Länge 4 oder 
	\item Kein Teilgraph $H$ ist ein Paar nicht inzidenter Kanten
\end{enumerate}

Wir führen folgende Fallunterscheidung durch:
\begin{description}
	\item[Fall 1] mit $|A|=4$ und $|B|=0$. Da $G[U]$ vollständig ist, ist auch jeder Teilgraph $H$ vollständig. Damit hat $H$ 6 Kanten und ist kein Kreis der Länge $4$ bzw. auch kein Paar nicht inzidenter Kanten.
	\item[Fall 2] mit $|A|=3$ und $|B|=1$. Da $G[U]$ vollständig ist, ist auch jeder Teilgraph $H$ in Bezug auf seine Knoten in $G[U]$ vollständig. Das heißt diese drei Knoten sind bereits alle miteinander verbunden. Wenn nun einer dieser Knoten $a_i$ ($i \in \{1,2,3\})$ mit dem Knoten $b_1$ verbunden ist, hat der Teilgraph $H$ vier Knoten, aber er ist kein Kreis. Wäre nun $b_1$ mit einem weiteren Knoten aus $A$ verbunden, wären es zu viel Knoten. Offensichtlich ist keiner der Kanten nicht paarweise inzident.
	\item[Fall 3] mit  $|A|=2$ und $|B|=2$. Der Teilgraph $H$ hat maximal drei Kanten, damit ist (a) erfüllt. Da $a_1$ und $a_2$ verbunden ist und $b_1$ und $b_2$ nur mit $a_1$ und $a_2$ verbunden sein können, ist (b) erfüllt.
	\item[Fall 4] mit  $|A|=1$ und $|B|=3$. Da $b_1, b_2$ und $b_3$ nicht mit einander verbunden sind, kann $H$ kein Kreis sein.  Da $b_1, b_2$ und $b_3$ maximal mit $a_1$ verbunden sind, ist (b) erfüllt.
	\item[Fall 5] mit  $|A|=0$ und $|B|=4$. Da die Knoten $b$ alle nicht miteinander verbunden sind, ist (a) und (b) erfüllt.
\end{description}

    	
    	
    	
    	
    	
    	
    	
    	
    	
    	
%    	Beweis per Induktion: \\
%    	\textbf{IA:} $n = 4$ \\ 
%    	Um einen Teilgraphen aus vier Knoten zu bauen, haben wir folgende drei Möglichkeiten:
%    	\begin{enumerate}
%    		\item Wir nehmen vier Knoten aus dem vollständigen Graphen;
%    		\item Wir nehmen einige Knoten aus dem vollständigen Graphen und einige aus unabhängige Knoten;
%    		\item Wir nehmen vier Knoten aus den unabhängigen Knoten.
%    	\end{enumerate}
%    	1. 
%    	\begin{tikzpicture}
%          	[scale=.8,auto=left,every node/.style={circle,fill=blue!20}]
%           	\node (n1) at (0,0) {};
%          	\node (n2) at (1,1) {};
%    		\node (n3) at (1,0) {};
%        	\node (n4) at (0,1) {};
%
%           	\foreach \from/\to in {n1/n2, n1/n3, n1/n4, n2/n3, n2/n4, n3/n4}
%            \draw (\from) -- (\to);
%        \end{tikzpicture}
%        \qquad 2.
%        \begin{tikzpicture}
%          	[scale=.8,auto=left,every node/.style={circle,fill=blue!20}]
%           	\node (n1) at (0,0) {};
%          	\node (n2) at (1,1) {};
%    		\node (n3) at (1,0) {};
%        	\node (n4) at (0,1) {};
%
%           	\foreach \from/\to in { n1/n3, n1/n4, n2/n4}
%            \draw (\from) -- (\to);
%        \end{tikzpicture}
%        \qquad 3.
%        \begin{tikzpicture}
%          	[scale=.8,auto=left,every node/.style={circle,fill=blue!20}]
%           	\node (n1) at (0,0) {};
%          	\node (n2) at (1,1) {};
%    		\node (n3) at (1,0) {};
%        	\node (n4) at (0,1) {};
%        \end{tikzpicture}\\\\
%        Daraus kann man deutlich erkennen, dass keine der drei Möglichkeiten ein Kreis oder ein Paar 
%        nicht inzidenter Kanten sein kann.\\\\
%        \textbf{IS:} von $n$ nach $n + 1$
%        Wir fügen zu jedem der drei Fälle einen Knoten hinzu. Dadurch erhalten wir:\\\\
%        1. 
%    	\begin{tikzpicture}
%          	[scale=.8,auto=left,every node/.style={circle,fill=blue!20}]
%           	\node (n1) at (0,0) {};
%          	\node (n2) at (2,2) {};
%    		\node (n3) at (2,0) {};
%        	\node (n4) at (0,2) {};
%        	\node (n5) at (1,1) {};
%
%           	\foreach \from/\to in {n1/n2, n1/n3, n1/n4, n2/n3, n2/n4, n3/n4}
%            \draw (\from) -- (\to);
%        \end{tikzpicture}
%        \qquad 2.
%        \begin{tikzpicture}
%          	[scale=.8,auto=left,every node/.style={circle,fill=blue!20}]
%           	\node (n1) at (0,0) {};
%          	\node (n2) at (2,2) {};
%    		\node (n3) at (2,0) {};
%        	\node (n4) at (0,2) {};
%        	\node (n5) at (1,1) {};
%
%           	\foreach \from/\to in {n1/n3, n1/n4, n2/n4, n5/n1, n5/n4}
%            \draw (\from) -- (\to);
%        \end{tikzpicture}
%        \qquad 3.
%        \begin{tikzpicture}
%          	[scale=.8,auto=left,every node/.style={circle,fill=blue!20}]
%    	    \node (n1) at (0,0) {};
%          	\node (n2) at (2,2) {};
%    		\node (n3) at (2,0) {};
%        	\node (n4) at (0,2) {};
%        	\node (n5) at (1,1) {};
%        \end{tikzpicture}\\\\
%        Da die Graphen aus dem vorherigen Beispiel Teilgraphen von diesem Beispiel sind, gilt das auch 
%        für eine andere Anzahl der Knoten. %das stimmt glaub net so ganz...
%        Sofern kann man sehen, dass induzierter Graph mit $V=4$ aus vollständiger Graph wieder ein 
%        vollständiger Graph ist und nach Definition des Kreises, einen Kreis bilden kann. Im zweiten 
%        Beispiel mit den unabhängigen Knoten, bleiben diese nicht verbunden und damit kann man auch keinen 
%        Kreis bilden. Im dritten Fall ist es offensichtlich, dass es kein Kreis sein kann, da die Kanten fehlen. 

    \section*{Selbststudium:}
        Erarbeiten Sie sich den Inhalt des Abschnitts "Bäume und Wälder" (Abschnitt 3.2) aus dem
        Skriptum \textit{Mathematik: Diskrete Strukturen} (Version v4.10 oder höher) und beantworten Sie
        folgende Fragen:
    \begin{enumerate}[label=(\alph*)]
        %Task (a)
        \item Ist jeder zusammenhängende Graph mit n Knoten und $n - 1$ Kanten eine Baum? Be-
        gründen Sie Ihre Antwort.\\
        Da wir $n-1$ Kanten haben, kann der Graph nicht gleichzeitig zusammenhängend sein und einen Kreis bilden, denn für einen Kreis in einem zusammenhängenden Graphen sind mindestens n Kanten erforderlich. Nach Theorem 3.13 gilt, das jeder Baum $n-1$ Kanten hat. Somit ist ein solcher Graph kreisfrei und erfüllt nach Definition 3.10 (MDS), die Eigenschaften eines Baumes. 

        %Task (b)
        \item Der Hyperwürfel $Q_3$ enthält 384 Spannbäume. %http://en.wikipedia.org/wiki/Hypercube_graph hier steht es drinn

        %Task (c)
        \item Welchen Prüfer-Code besitzt der folgende, markierte Baum?\\\\
        \begin{tikzpicture}
            [scale=.8,auto=left,every node/.style={circle,fill=blue!20}]
            \node (n2) at (1,1) {2};
            \node (n1) at (3,1) {1};
            \node (n7) at (5,1) {7};
            \node (n5) at (7,1) {5};
            \node (n4) at (3,-1) {4};
            \node (n6) at (2,-2) {6};
            \node (n8) at (4,-2) {8};
            \node (n3) at (5,-1) {3};
            \node (n9) at (6,-2) {9};

            \foreach \from/\to in {n2/n1, n1/n7, n7/n5, n1/n4, n4/n6, n4/n8, n7/n3, n3/n9}
            \draw (\from) -- (\to);
        \end{tikzpicture}\\\\
        Prüfer-Code: $1744173$

        %Task (d)
        \item Welcher markierte Baum hat den Prüfer-Code 212323212?\\\\
        \begin{tikzpicture}
            [scale=.8,auto=left,every node/.style={circle,fill=blue!20}]
            \node (n5) at (1,1) {5};
            \node (n10) at (3,1) {10};
            \node (n1) at (2,0) {1};
            
            \node (n11) at (1,-1) {11};
            \node (n7) at (5,-1) {7};
            
            \node (n4) at (1,-2) {4};
            \node (n2) at (2,-2) {2};
            \node (n3) at (4,-2) {3};
            
            \node (n6) at (1,-3) {6};
            \node (n8) at (2,-3) {8};
            \node (n9) at (5,-3) {9};

            \foreach \from/\to in {n5/n1, n10/n1, n1/n2, n2/n11, n2/n4, n2/n6, n2/n8, n2/n3, n3/n7, n9/n3}
            \draw (\from) -- (\to);
        \end{tikzpicture}

        \item Welcher Spannbaum des gegebenen Graphen besitzt den lexikographisch kleinsten Prüfer-Code?\\\\
        \begin{tikzpicture}
            [scale=.8,auto=left,every node/.style={circle,fill=blue!20}]
            \node (n0) at (1,0) {0};
            \node (n6) at (3,0) {6};
            
            \node (n1) at (0,-1) {1};
            \node (n2) at (2,-1) {2};
            \node (n4) at (3,-1) {4};
            
            \node (n5) at (2,-2) {5};
            \node (n7) at (3,-2) {7};

            \node (n3) at (2,-3) {3};

            \foreach \from/\to in {n0/n1, n0/n2, n0/n6, n6/n4, n5/n3, n4/n5, n4/n7}
            \draw (\from) -- (\to);
        \end{tikzpicture}\\
        Mit Prüfer-Code $006544$.
    \end{enumerate}
\end{document}