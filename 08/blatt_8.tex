% Math symbols and examples: http://en.wikibooks.org/wiki/LaTeX/Mathematics
% Also useful link: http://en.wikibooks.org/wiki/LaTeX/Advanced_Mathematics

% This document is used for title generating of exercise solutions. 
% Also it used for generating styles and environmentof documents.
% To add this title in document add next line:
% % This document is used for title generating of exercise solutions. 
% Also it used for generating styles and environmentof documents.
% To add this title in document add next line:
% % This document is used for title generating of exercise solutions. 
% Also it used for generating styles and environmentof documents.
% To add this title in document add next line:
% \input{../template.tex}
\documentclass[a4paper]{article}
\usepackage[utf8]{inputenc}
\usepackage[top=1.5cm]{geometry}
\usepackage{amssymb}
\usepackage{enumitem}
\usepackage{amsmath}

\allowdisplaybreaks

\DeclareMathOperator{\ggT}{ggT}
\DeclareMathOperator{\kgV}{kgV}

\title{Mathematik: Diskrete Strukturen \\ \Large Lösungsblatt}
\author{Anton Bubnov, Eugen Kuzmenko}

\documentclass[a4paper]{article}
\usepackage[utf8]{inputenc}
\usepackage[top=1.5cm]{geometry}
\usepackage{amssymb}
\usepackage{enumitem}
\usepackage{amsmath}

\allowdisplaybreaks

\DeclareMathOperator{\ggT}{ggT}
\DeclareMathOperator{\kgV}{kgV}

\title{Mathematik: Diskrete Strukturen \\ \Large Lösungsblatt}
\author{Anton Bubnov, Eugen Kuzmenko}

\documentclass[a4paper]{article}
\usepackage[utf8]{inputenc}
\usepackage[top=1.5cm]{geometry}
\usepackage{amssymb}
\usepackage{enumitem}
\usepackage{amsmath}

\allowdisplaybreaks

\DeclareMathOperator{\ggT}{ggT}
\DeclareMathOperator{\kgV}{kgV}

\title{Mathematik: Diskrete Strukturen \\ \Large Lösungsblatt}
\author{Anton Bubnov, Eugen Kuzmenko}

% Used to draw graphs. More info: http://en.wikipedia.org/wiki/PGF/TikZ
% Examples: http://www.texample.net/tikz/examples/
\usepackage{tikz} 

\begin{document}
    \maketitle
    \section*{Vertiefung:}
    \begin{enumerate}[label=(\alph*)]
        % Task (a)
        \item Welche Gradfolge besitzt $M_{3,4}$?\\
        Die Gradfolge von $M_{3,4}$ ist $4,4,3,3,3,3,3,3,2,2,2,2$.

        % Task (b)
        \item Gibt es einen Graphen mit Gradfolge $(5, 4, 3, 2, 2, 2, 2)$? - Und wenn ja, welchen?\\
        \begin{tikzpicture}
            [scale=.8,auto=left,every node/.style={circle,fill=blue!20}]
            \node (n1) at (3,1) {5};
            \node (n2) at (1,-1) {2};
            \node (n3) at (3,-1) {2};
            \node (n4) at (4,-1) {2};
            \node (n6) at (5,-1) {2};
            \node (n7) at (2,-2) {4};
            \node (n8) at (4,-2) {3};

            \foreach \from/\to in {n1/n2, n1/n3, n1/n4, n1/n6, n7/n1, n7/n2, n7/n3, n7/n8, n8/n6, n8/n4}
            \draw (\from) -- (\to);
        \end{tikzpicture}

        % Task (c)
        \item Gibt es einen Graphen mit Gradfolge $(5, 4, 3, 2, 2, 2, 2, 1)$? - Und wenn ja, welchen?\\
        Nach Proposition 3.3 gilt, dass die Summe der Grade über die Menge der Knoten eine gerade Zahl ergibt. 
        In unserem Fall beträgt die Summe der Grade 21, was eine ungerade Zahl ist. Folglich kann es 
        keinen solchen Graphen geben. 

        % Task (d)
        \item Gibt es einen Graphen mit Gradfolge $(5, 2, 1, 1, 1, 1, 1, 1, 1)$? - Und wenn ja, welchen?\\
        Es kann keinen solchen Graphen geben. Der Graph hätte 9 Knoten und (nach Proposition 3.3) 7 Kanten. 
        Folgende Graphik zeigt ein Beispiel für unseren Graphen: 
        \\\\
        \begin{tikzpicture}
            [scale=.8,auto=left,every node/.style={circle,fill=blue!20}]
            \node (n1) at (3,1) {5};
            \node (n2) at (1,-1) {1};
            \node (n3) at (2,-1) {1};
            \node (n4) at (3,-1) {1};
            \node (n5) at (4,-1) {1};
            \node (n6) at (5,-1) {1};
            \node (n7) at (6,-1) {1};
            \node (n8) at (7,-1) {1};

            \node (n9) at (6,1) {2};

            \foreach \from/\to in {n1/n2, n1/n3, n1/n4, n1/n5, n1/n6, n9/n8, n9/n7}
            \draw (\from) -- (\to);
        \end{tikzpicture}\\\\
        Es wird offensichtlich, dass es keine Möglichkeit zum Verbinden existiert. Ohne den Knoten mit 
        den 5 Verbindungen bleiben genau 2 Kanten übrig. Allgemein reichen 2 Kanten  nicht aus, um die 
        (verbliebenen) 8 Knoten zu verbinden. Ein Graph mit der Gradfolge $(5,2,1,1,1,1,1,1,1)$ kann es 
        folglich nicht geben. 
         
        % Task (e)
        \item Wie groß ist der maximale Abstand zweier Knoten im Hyperwürfel Q$_d$? \\
        Maximale Abstand zweier Knoten im Hyperwürfel Q$_d$ ist $d$.
        
        % Task (f)
        \item Wie groß ist der maximale Abstand zweier Knoten im Gittergraphen $M_{n,n}$ ? \\
        Maximale Abstand zweier Knoten im Gittergraphen $M_{n,n}$ ist $2n - 2$
        
        % Task (g)
        \item Welche Knoten haben im $M_{n,n}$ den kleinsten maximalen Abstand zu einem anderen Knoten? \\
        
        % Task (h)
        \item Welche Knoten haben im $M_{n,n}$ den größten maximalen Abstand zu einem anderen Knoten? \\
        
        % Task (i)
        \item Wie viele Wege der Länge $k$ enthält ein $r$-regulärer Graph mit $n$ Knoten? \\
        
        % Task (j)
        \item Wie viele Kreise der Länge $k$ enthält der vollständige Graph $K_n$? \\
        
        
    \end{enumerate}
    \section*{Kreativität:}
    \begin{enumerate}[label=(\alph*)]
        %Task (a)
        \item Task a
    \end{enumerate}
    \section*{Selbststudium:}
        Erarbeiten Sie sich den Inhalt des Abschnitts "Bäume und Wälder" (Abschnitt 3.2) aus dem
        Skriptum \textit{Mathematik: Diskrete Strukturen} (Version v4.10 oder höher) und beantworten Sie
        folgende Fragen:
    \begin{enumerate}[label=(\alph*)]
        %Task (a)
        \item Ist jeder zusammenhängende Graph mit n Knoten und $n - 1$ Kanten eine Baum? Be-
        gründen Sie Ihre Antwort.\\
        Jeder Knoten ist nach Voraussetzung über eine Kante verbunden. Da wir $n-1$ Kanten haben, 
        kann der Graph keinen Kreis bilden. Somit ist nach Definition 3.10 (MDS),
        da dieser Graph ist zusammenhangend und kreisfrei. %Lemma 3.11 und Lemma 3.12?
        \\ Außerdem gilt Theorem 3.13, da wir $n-1$ Kanten haben. Damit die Eigenschaft eines Baums erfüllt .

        %Task (b)
        \item %http://en.wikipedia.org/wiki/Hypercube_graph hier steht es drinn... aber warum das so ist ist eine gute Frage..

        %Task (c)
        \item Welchen Prüfer-Code besitzt der folgende, markierte Baum?\\\\
        \begin{tikzpicture}
            [scale=.8,auto=left,every node/.style={circle,fill=blue!20}]
            \node (n2) at (1,1) {2};
            \node (n1) at (3,1) {1};
            \node (n7) at (5,1) {7};
            \node (n5) at (7,1) {5};
            \node (n4) at (3,-1) {4};
            \node (n6) at (2,-2) {6};
            \node (n8) at (4,-2) {8};
            \node (n3) at (5,-1) {3};
            \node (n9) at (6,-2) {9};

            \foreach \from/\to in {n2/n1, n1/n7, n7/n5, n1/n4, n4/n6, n4/n8, n7/n3, n3/n9}
            \draw (\from) -- (\to);
        \end{tikzpicture}\\\\
        Prüfer-Code: $1744173$

        %Task (d)
        \item Welcher markierte Baum hat den Prüfer-Code 212323212?\\\\
        \begin{tikzpicture}
            [scale=.8,auto=left,every node/.style={circle,fill=blue!20}]
            \node (n5) at (1,1) {5};
            \node (n10) at (3,1) {10};
            \node (n1) at (2,0) {1};
            
            \node (n11) at (1,-1) {11};
            \node (n7) at (5,-1) {7};
            
            \node (n4) at (1,-2) {4};
            \node (n2) at (2,-2) {2};
            \node (n3) at (4,-2) {3};
            
            \node (n6) at (1,-3) {6};
            \node (n8) at (2,-3) {8};
            \node (n9) at (5,-3) {9};

            \foreach \from/\to in {n5/n1, n10/n1, n1/n2, n2/n11, n2/n4, n2/n6, n2/n8, n2/n3, n3/n7, n9/n3}
            \draw (\from) -- (\to);
        \end{tikzpicture}
    \end{enumerate}
\end{document}