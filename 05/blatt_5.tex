% Math symbols and examples: http://en.wikibooks.org/wiki/LaTeX/Mathematics
% Also useful link: http://en.wikibooks.org/wiki/LaTeX/Advanced_Mathematics

% This document is used for title generating of exercise solutions. 
% Also it used for generating styles and environmentof documents.
% To add this title in document add next line:
% % This document is used for title generating of exercise solutions. 
% Also it used for generating styles and environmentof documents.
% To add this title in document add next line:
% % This document is used for title generating of exercise solutions. 
% Also it used for generating styles and environmentof documents.
% To add this title in document add next line:
% \input{../template.tex}
\documentclass[a4paper]{article}
\usepackage[utf8]{inputenc}
\usepackage[top=1.5cm]{geometry}
\usepackage{amssymb}
\usepackage{enumitem}
\usepackage{amsmath}

\allowdisplaybreaks

\DeclareMathOperator{\ggT}{ggT}
\DeclareMathOperator{\kgV}{kgV}

\title{Mathematik: Diskrete Strukturen \\ \Large Lösungsblatt}
\author{Anton Bubnov, Eugen Kuzmenko}

\documentclass[a4paper]{article}
\usepackage[utf8]{inputenc}
\usepackage[top=1.5cm]{geometry}
\usepackage{amssymb}
\usepackage{enumitem}
\usepackage{amsmath}

\allowdisplaybreaks

\DeclareMathOperator{\ggT}{ggT}
\DeclareMathOperator{\kgV}{kgV}

\title{Mathematik: Diskrete Strukturen \\ \Large Lösungsblatt}
\author{Anton Bubnov, Eugen Kuzmenko}

\documentclass[a4paper]{article}
\usepackage[utf8]{inputenc}
\usepackage[top=1.5cm]{geometry}
\usepackage{amssymb}
\usepackage{enumitem}
\usepackage{amsmath}

\allowdisplaybreaks

\DeclareMathOperator{\ggT}{ggT}
\DeclareMathOperator{\kgV}{kgV}

\title{Mathematik: Diskrete Strukturen \\ \Large Lösungsblatt}
\author{Anton Bubnov, Eugen Kuzmenko}


\DeclareMathOperator{\ggT}{ggT}
\DeclareMathOperator{\kgV}{kgV}

\begin{document}
    \maketitle
    \section*{Vertiefung:}
    \begin{enumerate}[label=(\alph*)]
        % Task (a)
        \item Bei einem Turnier spielen n Mannschaften "jeder gegen jeden" (genau einmal). Für einen
		Sieg gibt es 3 Punkte, für ein Unentschieden 1 Punkt, für eine Niederlage 0 Punkte. Die
		Platzierungen ergeben sich aus den erzielten Punkten, bei Punktgleichheit wird gelost.
		Kann eine Mannschaft mit n Punkten Turnierletzter werden?
		
		% Task (b)
        \item Keine Antwort
        
        % Task (c)
        \item Wie viele Punkte kann der Turnierzweite höchstens erreichen, wenn das gleiche Szenario
		wie in Teilaufgabe (a) betrachtet wird?\\
		Turniererster kann maximal $(n-1)\cdot 3$ Punkten bekommen, wenn der in alle Spiele gewinnen 
		wird. Dann Turnierzweite einmal 
        
        % Task (d)
        \item Keine Antwort
        
        % Task (e)
        \item Keine Antwort
        
        % Task (f)
        \item Keine Antwort
        
        % Task (g)
        \item Keine Antwort
        
        % Task (h)
        \item Wie viele rekursive Aufrufe benötigt der Algorithmus von Euklid, um $\ggT(F_{k+2},F_{k+4})$
		zu bestimmen. Hierbei steht $F_k$ für die $k$-te Fibonacci-Zahl.
		Es gilt:
		\[\bmod(F_{k+4},F_{k+2}) = \bmod(F_{k+2} + F_{k+3},F_{k+2}) = \bmod(F_{k+3},F_{k+2}) = F_{k+1}\]
		Dann folgt:
		\begin{align*}
			\ggT(F_{k+2},F_{k+4}) &= EUKLID(F_{k+2},F_{k+4}) \\
			 &= EUKLID(\bmod(F_{k+4},F_{k+2}),F_{k+2}) &(1) \\ 
			 &= EUKLID(F_{k+1},F_{k+2}) &(2)
		\end{align*}
		Da die Schritt (1) ein Aufruf braucht und Schritt (2) braucht $k-1$ Aufrufe, haben wir insgesamt
		$k$ rekursive Aufrufe.

    \end{enumerate}
    \section*{Kreativität:}
    \begin{enumerate}[label=(\alph*)]
    	%Task (a)
    	\item Task a
    \end{enumerate}
    \section*{Transfer:}
    \begin{enumerate}[label=(\alph*)]
    	\item Task a
    \end{enumerate}
\end{document}






