% Math symbols and examples: http://en.wikibooks.org/wiki/LaTeX/Mathematics
% Also useful link: http://en.wikibooks.org/wiki/LaTeX/Advanced_Mathematics

% This document is used for title generating of exercise solutions. 
% Also it used for generating styles and environmentof documents.
% To add this title in document add next line:
% % This document is used for title generating of exercise solutions. 
% Also it used for generating styles and environmentof documents.
% To add this title in document add next line:
% % This document is used for title generating of exercise solutions. 
% Also it used for generating styles and environmentof documents.
% To add this title in document add next line:
% \input{../template.tex}
\documentclass[a4paper]{article}
\usepackage[utf8]{inputenc}
\usepackage[top=1.5cm]{geometry}
\usepackage{amssymb}
\usepackage{enumitem}
\usepackage{amsmath}

\allowdisplaybreaks

\DeclareMathOperator{\ggT}{ggT}
\DeclareMathOperator{\kgV}{kgV}

\title{Mathematik: Diskrete Strukturen \\ \Large Lösungsblatt}
\author{Anton Bubnov, Eugen Kuzmenko}

\documentclass[a4paper]{article}
\usepackage[utf8]{inputenc}
\usepackage[top=1.5cm]{geometry}
\usepackage{amssymb}
\usepackage{enumitem}
\usepackage{amsmath}

\allowdisplaybreaks

\DeclareMathOperator{\ggT}{ggT}
\DeclareMathOperator{\kgV}{kgV}

\title{Mathematik: Diskrete Strukturen \\ \Large Lösungsblatt}
\author{Anton Bubnov, Eugen Kuzmenko}

\documentclass[a4paper]{article}
\usepackage[utf8]{inputenc}
\usepackage[top=1.5cm]{geometry}
\usepackage{amssymb}
\usepackage{enumitem}
\usepackage{amsmath}

\allowdisplaybreaks

\DeclareMathOperator{\ggT}{ggT}
\DeclareMathOperator{\kgV}{kgV}

\title{Mathematik: Diskrete Strukturen \\ \Large Lösungsblatt}
\author{Anton Bubnov, Eugen Kuzmenko}


\begin{document}
	\maketitle
	\section*{Vertiefung:}
	\begin{enumerate}[label=(\alph*)]
		% Task (a)
		\item  Bestimmen Sie $ \bmod(5^{31} \cdot 2^{789}-23^{23}, 10) $.
		\begin{align*}
			% part 1
			\bmod(5^{31},10) &= \bmod(5^{30}\cdot5,10) 
			\\&= 5 \\\\
			% part 2
			\bmod(2^{789},10) &= \bmod(2^{516}\cdot2^{256}\cdot2^{16}\cdot2^1,10) 
			\\&= \bmod(6\cdot 6\cdot 6\cdot 2, 10) 
			\\&= \bmod(432,10) 
			\\&= 2 \\\\
			% part 3
			\bmod(-23^{23},10) &= \bmod(23^{16}\cdot23^{4}\cdot23^{2}\cdot(-23)^1,11) 
			\\&= \bmod(9\cdot 7, 10) 
			\\&= \bmod(63,10) 
			\\&= 3 \\\\
			% result
			\bmod(5^{31} \cdot 2^{789}-23^{23}, 10) &= \bmod(5 \cdot 2 + 3,10)	
			\\&= \bmod(13,10) 
			\\&= 3
			% comment
		 	&\textrm{(nach Theorem 1.2 (BM))}
		\end{align*}

		% Task (b)
		\item  Bestimmen Sie $ \bmod(5^{31} \cdot 2^{789}-23^{23}, 11) $.
		\begin{align*}
			% part 1
			\bmod(5^{31},11) &= \bmod(5^{30}\cdot5,11)
			\\&= 5 \\\\
			% part 2
			\bmod(2^{789},11) &= \bmod(2^{516}\cdot2^{256}\cdot2^{16}\cdot2^1,11)
			\\&= \bmod(9\cdot 9\cdot 9\cdot 2, 11)
			\\&= \bmod(432,11)
			\\&= 6 \\\\
			% part 3
			\bmod(-23^{23},11) &= \bmod(23^{16}\cdot23^{4}\cdot23^{2}\cdot(-23)^1,11)
			\\&= \bmod(4\cdot 3\cdot 5\cdot 7, 11)
			\\&= \bmod(420,11)
			\\&= 9 \\\\ 
			% result
			\bmod(5^{31} \cdot 2^{789}-23^{23}, 11) &= \bmod(5 \cdot 6 + 9,11)
			\\&= \bmod(39,11)
			\\&= 6 
			% comment
		 	&\textrm{(nach Theorem 1.2 (BM))}
		\end{align*}

		% Task (c)
		\item  Bestimmen Sie $ \bmod(7^{31} \cdot 2^{789}, 10) $.
		\begin{align*}
			% part 1
			\bmod(7^{31},10) &= \bmod(7^{16}\cdot7^{8}\cdot7^{4}\cdot7^{2}\cdot7^{1},10)
			\\&= \bmod(1\cdot1\cdot1\cdot9\cdot7,10)
			\\&= \bmod(63,10)
			\\&= 3 \\\\
			% part 2
			\bmod(2^{789},10) &= \bmod(2^{516}\cdot2^{256}\cdot2^{16}\cdot2^1,10)
			\\&= \bmod(6\cdot 6\cdot 6\cdot 2, 10)
			\\&= \bmod(432,10)
			\\&= 2 \\\\
			% result
			\bmod(7^{31} \cdot 2^{789}, 10) &= \bmod(3 \cdot 2,10)
			\\&= \bmod(6,10)
			\\&= 6 
			% comment
		 	&\textrm{(nach Theorem 1.2 (BM))}
		\end{align*}

		% Task (d)
		\item Bestimmen Sie kgV(178, 144).
		\begin{align*}
			& 178 = 2\cdot89 \\
			& 144 = 2^4\cdot3^2 \\
			&\kgV(178, 144) = 2^4\cdot3^2\cdot89 = 12816
			% comment
		 	&\textrm{(nach Lemma 1.5 (BM))}
		\end{align*}

		% Task (e)
		\item Bestimmen Sie ggT(12877480, 24145275).
		\begin{align*}
			\ggT(12877480, 24145275) &= \ggT(24145275-12877480, 12877480)
			\\& = \ggT(12877480-11267795,11267795)
			\\& = \ggT(11267795-1609685,1609685)
			\\& = 1609685
			% comment
		 	&\textrm{(nach Lemma 1.8 (BM))}
		\end{align*}

		% Task (f)
		\item Wie sieht der $\frac{12877480}{24145275}$ zu äquivalente teilerfremde Bruch aus?
		\begin{align*}
		\end{align*}

		% Task (g)
		\item Wie viele Funktionen $f : \{0, 1, 2, 3\}^n \to \{0, 1, 2\}$ gibt es, die genau einmal den Funktionswert 0 annehmen?
		\begin{align*}
			\intertext{
				F\"ur erste Stelle nehmen wir 0. Dann f\"ur die Reste (3 Stellen) sind nur zwei Kugeln \{1,2\} m\"oglich.
				Dann haben wir:
			} 
			1\cdot2\cdot2\cdot2 = 8
			\intertext{
				Also wenn 0 an die erste Stelle ist, haben wir 8 Varianten. 
				Da es ist auch M\"oglich 0 an zweite, dritte und vierte Stelle stellen, haben wir:
			}
			8\cdot4 = 32
		\end{align*}

		% Task (h)
		\item Wie viele Funktionen $f : \{0, 1, 2, 3\}^n \to \{0, 1, 2\}$ gibt es, die genau zweimal den Funktionswert 0 annehmen?
		\begin{align*}
			\intertext{
				Es gibt 6 Moeglichkeiten zwei 0 an 4 Stellen zu stellen. Da fuer jede diese Moeglichkeit gibt es auch 4 Moeglichkeiten \{1,2\} zu stellen. Dan haben wir:
				%In diesem Fall wir nehmen 4 Kugeln aus \{0,1,2\}. Da wir ungeordnete Reihe ohne zur\"uckliegen brauchen, 
				%benutzen wir Kombination Formel (Theorem 1.6 (MDS)) 
			}
			6\cdot4 = 24
		\end{align*}
		
	\end{enumerate} 
	
	\section*{Transfer:}
	\begin{enumerate}[label=(\alph*)]
		\item 8 Symbole
		\begin{enumerate}[label=\bfseries Schritt \arabic*:]
			% Schritt 1
			\item Wir w\"ahlen beliebig zwei S\"atze. Um L\"ange 8 zu bekommen mu\"ssen wir insgesamt 4 W\"ortern haben,
			weil andere 4 Symbolen zus\"atzliche Zeichen sind.

			% Schritt 2
			\item Da es in Alphabet 26 Buchstaben gibt, k\"onnen wir von 4 kleine Buchstaben so viel Kombinationen machen:
			\begin{align*}
				26^4=456976
			\end{align*}

			% Schritt 3
			\item Wenn jede kleine Buchstabe eventuell auch gr\"osse sein kann, k\"onnen wir von 4 kleine und grosse Buchstaben so viel Kombinationen machen:
			\begin{align*}
				(26\cdot2)^4=7311616
			\end{align*}

			% Schritt 4
			\item In Reihe von 4 Buchstaben k\"onnen wir Ziffern in 7 Varianten stellen:
			\begin{align*}
				& X \in {0,1,...,9} \\
				& \{a|bcd\} \to \{Xa|Xbcd\},\{aX|bcdX\} \\
				& \{ab|cd\} \to \{Xab|Xcd\},\{aXb|cXd\}, \{abX|cdX\} \\
				& \{abc|d\} \to \{Xabc|Xd\},\{abcX|dX\}
			\end{align*}
			Erster Zahl kann 10 Ziffer sein und zweite 9 Ziffern. Also haben wir so viele Varianten f\"ur 2 Verschiedene 
			Ziffern und 7 Satz Kombinazionen:
			\begin{align*}
				& 10\cdot9\cdot7 = 630 \\
				& 7311616\cdot630 = 4606318080
			\end{align*}

			% Schritt 5
			\item N\"achstens haben wir 3 M\"oglichkeiten 6 Symbolen zu stellen:
			\begin{align*}
				& 3\cdot6=18 \\
				& 4606318080\cdot18 = 82913725440
			\end{align*}

			% Schritt 6
			\item Dann haben wir 22 Symbole am Anfang oder am Ende:
			\begin{align*}
				& 22\cdot2=44 \\
				& 82913725440\cdot44 = 3648203919360
			\end{align*}

		\end{enumerate}
	\end{enumerate}
\end{document}