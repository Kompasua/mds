% Math symbols and examples: http://en.wikibooks.org/wiki/LaTeX/Mathematics
% Also useful link: http://en.wikibooks.org/wiki/LaTeX/Advanced_Mathematics

% This document is used for title generating of exercise solutions. 
% Also it used for generating styles and environmentof documents.
% To add this title in document add next line:
% % This document is used for title generating of exercise solutions. 
% Also it used for generating styles and environmentof documents.
% To add this title in document add next line:
% % This document is used for title generating of exercise solutions. 
% Also it used for generating styles and environmentof documents.
% To add this title in document add next line:
% \input{../template.tex}
\documentclass[a4paper]{article}
\usepackage[utf8]{inputenc}
\usepackage[top=1.5cm]{geometry}
\usepackage{amssymb}
\usepackage{enumitem}
\usepackage{amsmath}

\allowdisplaybreaks

\DeclareMathOperator{\ggT}{ggT}
\DeclareMathOperator{\kgV}{kgV}

\title{Mathematik: Diskrete Strukturen \\ \Large Lösungsblatt}
\author{Anton Bubnov, Eugen Kuzmenko}

\documentclass[a4paper]{article}
\usepackage[utf8]{inputenc}
\usepackage[top=1.5cm]{geometry}
\usepackage{amssymb}
\usepackage{enumitem}
\usepackage{amsmath}

\allowdisplaybreaks

\DeclareMathOperator{\ggT}{ggT}
\DeclareMathOperator{\kgV}{kgV}

\title{Mathematik: Diskrete Strukturen \\ \Large Lösungsblatt}
\author{Anton Bubnov, Eugen Kuzmenko}

\documentclass[a4paper]{article}
\usepackage[utf8]{inputenc}
\usepackage[top=1.5cm]{geometry}
\usepackage{amssymb}
\usepackage{enumitem}
\usepackage{amsmath}

\allowdisplaybreaks

\DeclareMathOperator{\ggT}{ggT}
\DeclareMathOperator{\kgV}{kgV}

\title{Mathematik: Diskrete Strukturen \\ \Large Lösungsblatt}
\author{Anton Bubnov, Eugen Kuzmenko}

\DeclareRobustCommand{\stirling}{\genfrac\{\}{0pt}{}}
\begin{document}
    \maketitle
    \section*{Vertiefung:}
    \begin{enumerate}[label=(\alph*)]
        % Task (a)
        \item Drücken Sie die Anzahl der surjektiven Funktionen 
        $f : \{0, 1\}^n \to \{0, 1\}^2$ mit Hilfe der Stirling-Zahlen zweiter Art aus.\\\\
        Nach Lemma 4 (Potenzregel) und Kreuzprodukt Definition es gilt entsprechend:
        \begin{align*}
	        &||\{0, 1\}^n|| = 2^n \\
	        &|| \{0, 1\}^2|| = 2^2 = 4
	    \end{align*}
	    Wir muessen $2^n$ Funktionsargumente auf $4$ Funktionswerte abbilden. Da Stirling-Zahlen auf 
	    nicht unterscheidbare Funktionswerte aufzuteilt, sollen wir noch nit $4!$ multiplizieren.  
	    Folglich: \[4!\cdot\stirling{2^n}{4}\]

        % Task (b)
        \item Kein Antwort

        % Task (c)
        \item Von 18 Studierenden in einer Spezialvorlesung studieren 7 Mathematik, 9 Physik und 10
		Informatik. Davon studieren 3 Mathematik und Physik, 3 Mathematik und Informatik
		sowie 5 Physik und Informatik. Ein Student studiert sogar all drei Fächer. Wie viele
		Studierende studieren keines der drei Fächer?\\
		Sei nach Voraussetzung:
		\[||M|| = 7,\, ||I|| = 10,\, ||P|| = 9,\, ||M \cap P|| = 3,\, ||M \cap I|| = 3,\, 
		  ||P \cap I|| = 5,\, ||M \cap P \cap I|| = 1\]
		Die gesamte Zahl der Studierenden, die in einer Spezialvorlesung studieren ist:
		\[||M \cup P \cup I||\]
		Nach Theorem 1.19 Beispiel es gilt:
		\begin{align*}
	        ||M\cup P\cup I|| &= ||M||+||P||+||I||-||M\cap P||-||M\cap I||-||I \cap P||+||M \cap P \cap I||\\
	        &= 7 + 9 + 10 - 3 - 3 - 5 + 1 \\
	        &= 16
        \end{align*}
        Folglich die Anzahl der Studierende, die keines der drei Fächer studieren ist $18-16 = 2$.

        % Task (d)
        \item Für drei Mengen $A, B$ und $C$ gelten folgende Eigenschaften: 
        $||A|| = 63, ||B|| = 91, ||C|| = 44, ||A\cap B|| = 25, ||A\cap C|| = 23, ||C \cap B|| = 21$. 
        Außerdem gelte $||A\cup B\cup C|| = 139$. Wie groß ist $||A \cap B \cap C||$?\\\\
        Nach Theorem 1.19 Beispiel es gilt:
        \begin{align*}
	        ||A\cup B\cup C|| &= ||A||+||B||+||C||-||A\cap B||-||A\cap C||-||C \cap B||+||A \cap B \cap C||\\
	        \Rightarrow ||A \cap B \cap C|| &= ||A\cup B\cup C||-||A||-||B||-||C||+||A\cap B||+||A\cap C||+||C \cap B||\\
	        \Rightarrow ||A \cap B \cap C|| &= 139 - 63 - 91 - 44 + 25 + 23 + 21 = 10 \\
        \end{align*}

        % Task (e)
        \item Für zwei Mengen A und B gelte: $||A|| = 100, ||B|| = 60$ und die Anzahl 
        der Elemente von A  B, die zu genau einer der beiden Mengen gehören, ist genau 
        doppelt so groß, wie die Anzahl der Elemente, die in beiden Mengen liegen. Wie viele Elemente liegen in beiden Mengen?
        \begin{align*}
        	||A \Delta B|| &= 2 \cdot ||A \cap B|| \\
        	||A \cup B || &= ||A \Delta B|| + ||A \cap B|| \\
        	||A \cup B || &= ||A|| + ||B|| - ||A \cap B|| \\
        	||A \cap B || &= ||A|| + ||B|| - ||A \cup B|| \\
        	||A \cap B || &= ||A|| + ||B|| - 3||A \cap B|| \\
        	4 \cdot ||A \cup B || &= ||A|| + ||B|| \\
        	||A \cap B|| &= \frac{||A||+||B||}{4}= \frac{100 + 60}{4} = 40 
        \end{align*}
        
        % Task (f)
        \item Wie viele Zahlen im Bereich $1, 2,\ldots, 200$ sind durch keine der Zahlen 3, 7, 11, 27 teilbar?\\
        \begin{align*}
	        &200 - \bigg(\bigg\lfloor\frac{200}{3}\bigg\rfloor + \bigg\lfloor\frac{200}{7}\bigg\rfloor
	         + \bigg\lfloor\frac{200}{11}\bigg\rfloor + \bigg\lfloor\frac{200}{27}\bigg\rfloor \bigg) \\
	         &+  \bigg(\bigg\lfloor\frac{200}{21}\bigg\rfloor + \bigg\lfloor\frac{200}{33}\bigg\rfloor 
	         + \bigg\lfloor\frac{200}{81}\bigg\rfloor + \bigg\lfloor\frac{200}{77}\bigg\rfloor
	         + \bigg\lfloor\frac{200}{189}\bigg\rfloor\bigg) \\ 
	         &-\bigg(\bigg\lfloor\frac{200}{3\cdot 7\cdot 11}
	         \bigg\rfloor+\bigg\lfloor\frac{200}{3\cdot 7\cdot 27}\bigg\rfloor + \bigg\lfloor\frac{200}{7\cdot 11\cdot 27}\bigg\rfloor +
	         \bigg\lfloor\frac{200}{3\cdot 11\cdot 27}\bigg\rfloor \bigg)  \\
	         &+ \bigg\lfloor\frac{200}{3\cdot 7\cdot 11 \cdot 27}\bigg\rfloor \\
	         &= 200 - (66+28+18+7) + (9 + 6 + 2 + 2 + 1) - (0 + 0 + 0 + 0) + 0 = 101
        \end{align*}
        % Task (g)
        \item Wie viele Zahlen im Bereich $1,...,10^9$ sind weder von der Form $x^3$ noch $x^7$ noch $x^{13}$
        für ein geeignetes $x \in N $?\\
         Es gibt: 
	    \begin{align*}
	    &\lfloor 10^{9/3}\rfloor = 1000  \text{ Zahlen der Form } x^3 \\
	    &\lfloor 10^{9/7}\rfloor= 19  \text{ Zahlen der Form }x^7 \\
	    &\lfloor 10^{9/13}\rfloor= 4  \text{ Zahlen der Form }x^{13} \\
	    &\lfloor 10^{9/21}\rfloor= 2  \text{ Zahlen der Form } (x^{3})^7 = x^{21} \\
	    &\lfloor 10^{9/39}\rfloor= 1  \text{ Zahlen der Form } (x^{3})^{13} = x^{39} \\
	    &\lfloor 10^{9/91}\rfloor= 1  \text{ Zahlen der Form } (x^{7})^{13} = x^{91} \\
	    &\lfloor 10^{9/273}\rfloor= 1  \text{ Zahlen der Form } ((x^{3})^7)^{13} = x^{273} \\
	    \\
	    10^9-(1000+19+4)&+(2+1+1)-1=999\; 998\; 980  
        \end{align*}

        % Task (i)
        \item Menschen haben bis zu 150.000 Kopfhaare. Mindestens wie viele Chinesen haben die
		exakt gleiche Anzahl von Kopfhaaren (zu einem bestimmten Zeitpunkt), wenn Sie von
		einer chinesischen Bevölkerung von 1,33 Millarden ausgehen?\\\\
		\textbf{Bemerkung}: Wir vermuten, dass alle Chinesen mindestens 1 Haar haben.\\
		Verwenden wir ein Schubfachprinzip. Dann gilt:
		\[\bigg\lceil\frac{1330000000}{150000}\bigg\rceil = 8867\]
    \end{enumerate}
    \section*{Kreativit\"at:}
    \begin{enumerate}[label=(\alph*)]
    	\item Zeigen Sie, dass es eine Stadt geben muss, von der man direkt in mindestens die Hälfte
		aller anderen Städte kommen kann.\\\\
		Sei $k$ Anzahl der Knoten und $n$ maximale Anzahl der Kanten, die ein Knoten haben kann. 
		Es ist so, dass erster (beliebiger) Knoten hat $n_k = k-1$, dann nechster ein weniger, 
		also $k-2$ usw. bis $n_1 = 1$. 
		Die gesamte Anzahl der Kanten k\"onnen wir so schreiben (Arithmetische Folge): 
		\[S_n = \frac{a_1+a_n}{2}\cdot n = \frac{1+n}{2}\cdot n\]
		Verwenden wir f\"ur Beweis ein Schubfachprinzip. Wir sollen $S_n$ orientirete Kanten auf $k$ 
		Knoten stellen. Nach Schubfachprinzip und Voraussetzung es gilt: 
		\begin{align*}
			&\bigg\lceil\frac{S_n}{k}\bigg\rceil \geq \frac{k}{2} \\
			\Rightarrow &\bigg\lceil\frac{\frac{1+n}{2}\cdot n}{n+1}\bigg\rceil \geq \frac{n+1}{2}\\
			\Rightarrow &\bigg\lceil\frac{(n+1)\cdot n}{2} \cdot \frac{1}{n+1}\bigg\rceil \geq \frac{n+1}{2}\\
			\Rightarrow &\bigg\lceil\frac{n}{2}\bigg\rceil \geq \frac{n+1}{2}\\
			\Rightarrow &\bigg\lfloor\frac{n}{2}\bigg\rfloor + 1 \geq \frac{n+1}{2}\\
			\Rightarrow &\bigg\lfloor\frac{n}{2}\bigg\rfloor \geq \frac{n+1}{2} - 1\\
			\Rightarrow &\bigg\lfloor\frac{n}{2}\bigg\rfloor \geq \frac{n-1}{2}\\
		\end{align*}
		Da die letze Aussage gilt, die Voraussetzung ist bewiesen.
    \end{enumerate}
\end{document}






