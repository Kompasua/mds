% Math symbols and examples: http://en.wikibooks.org/wiki/LaTeX/Mathematics
% Also useful link: http://en.wikibooks.org/wiki/LaTeX/Advanced_Mathematics

% This document is used for title generating of exercise solutions. 
% Also it used for generating styles and environmentof documents.
% To add this title in document add next line:
% % This document is used for title generating of exercise solutions. 
% Also it used for generating styles and environmentof documents.
% To add this title in document add next line:
% % This document is used for title generating of exercise solutions. 
% Also it used for generating styles and environmentof documents.
% To add this title in document add next line:
% \input{../template.tex}
\documentclass[a4paper]{article}
\usepackage[utf8]{inputenc}
\usepackage[top=1.5cm]{geometry}
\usepackage{amssymb}
\usepackage{enumitem}
\usepackage{amsmath}
\usepackage{fancyhdr}


\allowdisplaybreaks

\DeclareMathOperator{\ggT}{ggT}
\DeclareMathOperator{\kgV}{kgV}

\pagestyle{fancy}
\fancyhf{}
\rhead{Anton Bubnov, Yevgen Kuzmenko}
\lhead{Mathematik: Diskrete Strukturen}
\cfoot{\thepage}

\title{Mathematik: Diskrete Strukturen \\ \Large Lösungsblatt}
\author{Anton Bubnov, Yevgen Kuzmenko}

\documentclass[a4paper]{article}
\usepackage[utf8]{inputenc}
\usepackage[top=1.5cm]{geometry}
\usepackage{amssymb}
\usepackage{enumitem}
\usepackage{amsmath}
\usepackage{fancyhdr}


\allowdisplaybreaks

\DeclareMathOperator{\ggT}{ggT}
\DeclareMathOperator{\kgV}{kgV}

\pagestyle{fancy}
\fancyhf{}
\rhead{Anton Bubnov, Yevgen Kuzmenko}
\lhead{Mathematik: Diskrete Strukturen}
\cfoot{\thepage}

\title{Mathematik: Diskrete Strukturen \\ \Large Lösungsblatt}
\author{Anton Bubnov, Yevgen Kuzmenko}

\documentclass[a4paper]{article}
\usepackage[utf8]{inputenc}
\usepackage[top=1.5cm]{geometry}
\usepackage{amssymb}
\usepackage{enumitem}
\usepackage{amsmath}
\usepackage{fancyhdr}


\allowdisplaybreaks

\DeclareMathOperator{\ggT}{ggT}
\DeclareMathOperator{\kgV}{kgV}

\pagestyle{fancy}
\fancyhf{}
\rhead{Anton Bubnov, Yevgen Kuzmenko}
\lhead{Mathematik: Diskrete Strukturen}
\cfoot{\thepage}

\title{Mathematik: Diskrete Strukturen \\ \Large Lösungsblatt}
\author{Anton Bubnov, Yevgen Kuzmenko}

\DeclareMathOperator{\kgV}{kgV}

\begin{document}
    \maketitle
    \section*{Vertiefung:}
    \begin{enumerate}[label=(\alph*)]
        % Task (a)
        \item Ist $\langle \mathbb{R}_{>0} , f \rangle$ mit der Verknüpfung $f : (x,y) \mapsto  \log_y x$ eine Algebra? \\
        Ja, denn $\mathbb{R}_{>0}$ stellt eine nichtleere Menge dar. Der Operator $f$ ist zudem eine Abbildung, 
        der Logarithmus, der für alle $\mathbb{R}_{>0}$ gültig ist. Definition 4.1 ist somit erfüllt.
        
        % Task (b)
        \item Ist $\langle \{n| n \in \mathbb{N} \textrm{ ist eine ungerade Zahl} \}, f \rangle$ mit dem dreistelligen 
        Operator $f : (x,y,z) \mapsto \bmod(x^2 + 2y, z + 2)$ eine Algebra? \\
        Nein. Betrachte hierzu den Fall $x = 1, y = 1, z = 1$. In diesem Fall würde gelten 
        $$\bmod(1^2 + 2 \cdot 1, 1 + 2 ) = \bmod(3,3) = 0$$ Die Null ist aber eine gerade Zahl und muss auch 
        in $n$ liegen. Aus diesem Widerspruch folgt, dass hier keine Algebra vorliegt.
        
        % Task (c)
        \item
        
        % Task (d)
        \item 
        
        % Task (e)
        \item 
        
        % Task (f)
        \item Wie viele inverse Elemente besitzt 2 in der Algebra $\langle \mathbb{N}_+ , \kgV \rangle$?\\
        Überhaupt kein einziges.
        
        % Task (g)
        \item 
        
        % Task (h)
        \item Können Sie die gegebene Verknüpfungstabelle für $\circ$ so ergänzen, dass $\circ$ assoziativ 
        auf \{a, b, c, d\} ist?\\
        Das geht nicht auf.
        
        % Task (i)
        \item
        
        % Task (j)
        \item         
        
    \end{enumerate}
    \section*{Kreativität:}
    \begin{enumerate}[label=(\alph*)]
    	%Task (a)
    	\item Task a
    \end{enumerate}
    \section*{Transfer:}
    \begin{enumerate}[label=(\alph*)]
    	\item Task a
    \end{enumerate}
\end{document}






