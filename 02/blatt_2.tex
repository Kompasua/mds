% Math symbols and examples: http://en.wikibooks.org/wiki/LaTeX/Mathematics
% Also useful link: http://en.wikibooks.org/wiki/LaTeX/Advanced_Mathematics

% This document is used for title generating of exercise solutions. 
% Also it used for generating styles and environmentof documents.
% To add this title in document add next line:
% % This document is used for title generating of exercise solutions. 
% Also it used for generating styles and environmentof documents.
% To add this title in document add next line:
% % This document is used for title generating of exercise solutions. 
% Also it used for generating styles and environmentof documents.
% To add this title in document add next line:
% \input{../template.tex}
\documentclass[a4paper]{article}
\usepackage[utf8]{inputenc}
\usepackage[top=1.5cm]{geometry}
\usepackage{amssymb}
\usepackage{enumitem}
\usepackage{amsmath}

\allowdisplaybreaks

\DeclareMathOperator{\ggT}{ggT}
\DeclareMathOperator{\kgV}{kgV}

\title{Mathematik: Diskrete Strukturen \\ \Large Lösungsblatt}
\author{Anton Bubnov, Eugen Kuzmenko}

\documentclass[a4paper]{article}
\usepackage[utf8]{inputenc}
\usepackage[top=1.5cm]{geometry}
\usepackage{amssymb}
\usepackage{enumitem}
\usepackage{amsmath}

\allowdisplaybreaks

\DeclareMathOperator{\ggT}{ggT}
\DeclareMathOperator{\kgV}{kgV}

\title{Mathematik: Diskrete Strukturen \\ \Large Lösungsblatt}
\author{Anton Bubnov, Eugen Kuzmenko}

\documentclass[a4paper]{article}
\usepackage[utf8]{inputenc}
\usepackage[top=1.5cm]{geometry}
\usepackage{amssymb}
\usepackage{enumitem}
\usepackage{amsmath}

\allowdisplaybreaks

\DeclareMathOperator{\ggT}{ggT}
\DeclareMathOperator{\kgV}{kgV}

\title{Mathematik: Diskrete Strukturen \\ \Large Lösungsblatt}
\author{Anton Bubnov, Eugen Kuzmenko}


\begin{document}
    \maketitle
    \section*{Vertiefung:}
    \begin{enumerate}[label=(\alph*)]
        % Task (a)
        \item  Wie viele Verlosungen von 5 identischen Kaffeemaschinen unter 25 Teilnehmern gibt es?\\\\
        Ziehen ohne Zurücklegen, ohne Reihenfolge.\\    
        $${n \choose k}= {25 \choose 5}= \frac{25!}{5!(25-5)!}= 
        \frac{20!\cdot 21 \cdot 22 \cdot 23 \cdot 24 \cdot 25}{5! \cdot 20!} = 53 \; 1230 $$
        
        % Task (b)
        \item Wie viele Möglichkeit gibt es, genau 7 Chips auf die drei Felder 1-12, 13-24, 25-36 
        beim Roulette zu legen?\\\\ 
        Wir ziehen die Felder ohne Reihenfolge mit Zur\"ucklegen.
        $n=3; \; k=7$ \\
        $$ \binom{n+k-1}{k}= \binom{3+7-1}{7}= \binom{9}{7}=
        \frac{9!}{7!(2!)}= \frac{8*9}{2}=36 $$

        % Task (c)
        \item Wie viele Binärzahlen der Länge 8 beginnen mit einer 0 oder enden mit 11?\\\\
        Da am Anfang haben wir 0, bleib f\"ur die Reste 7 Stellen, also $2^7 = 128$ Varianten.\\
        Da am Ende haben wir zwei 1, bleib f\"ur die Reste 6 Stellen, also $2^6 = 64$ Varianten.\\
        Nach Satz 6. (geordnet, mit Zur\"ucklegen)

        % Task (d)
        \item Sie haben 5 Informatik-Bücher, 4 Mathematik-Bücher und 3 Philosophie-Bücher zur
        Auswahl. Wie viele Möglichkeiten gibt es, auf eine Reise zwei Bücher aus verschiedenen
        Themenbereichen mitzunehmen?\\\\
        Nach Satz 6. (ungeordnet, ohne Zur\"ucklegen) es gibt 3 Kombinationen von F\"achern und zwar
        Inf. und Mathe, Inf. und Phil., Mathe und Phil.\\
        Gemäß jeden Fall haben wir insgesamt so viel Kombinationen:
        \[5 \cdot 4 + 5 \cdot 3 + 4 \cdot 3 = 47\]

        % Task (e)
        \item In einer Gruppe von 11 Personen sind 5 Vorstandsposten (ohne Personalunion) für jeweils
        eine Person zu vergeben: Präsidentin, Vizepräsidentin, Geschäftsführerin, stellvertreten-
        de Geschäftsführerin, Schatzmeisterin. Wie viele verschiedene Vorstände sind möglich?\\\\
        In diesem Fall sollen wir ein Formel aus Satz 6. (geordnet, ohne Zur\"ucklegen) benutzen:
        \[\frac{n!}{(n-k)!}=\frac{11!}{(11-5)!}
        =\frac{6!\cdot7\cdot8\cdot9\cdot10\cdot11}{6!}
        =55440\]

        % Task (f)
        \item Wie viele verschiedene Wörter können Sie aus dem Wort PICHICHI bilden?\\\\
        In dieser Wort haben wir 8 Buchstaben. Da aber manche Buchstaben sich viederholen: 
        I=3, C=2, H=2 - Mal. Da wir die Umstellung der gleiche Buchstaben nicht unterscheiden, haben wir:
        \[\frac{n!}{k_1!\cdot k_2!\cdot\ldots\cdot k_n!} = \frac{8!}{3!\cdot2!\cdot2!} = 1680\]
        Bemerkung: $k_n!$ ist eine Variation von Buchstaben Umstellung, die sich viederholen. 
        Da wir solche Woertern nicht akzeptiren sollen, ausschließen wir ihre aus gesamte Variationen. 

        % Task (g)
        \item Welcher Faktor B(n, k) erfüllt die Gleichung $\binom{n}{k} = B(n,k)\cdot\binom{n-1}{k-1}$?
        \begin{align*}
        \binom{n}{k} &=_{def} \frac{n!}{k!\cdot(n-k)!}\\
        B(n,k) &= \frac{\binom{n}{k}}{\binom{n-1}{k-1}} &\textrm{(Nach Voraussetzung)}\\
        &= \frac{n!}{k!\cdot(n-k)!} \cdot \frac{(k-1)!\cdot(n-k)!}{(n-1)!} &\textrm{(Nach Definition)}\\
        &= \frac{n\cdot(n-1)!}{k\cdot(k-1)!} \cdot \frac{(k-1)!\cdot(n-k)!}{(n-1)!}\\
        &= \frac{n}{k}
        \end{align*}
        
        % Task (h)
        \item Welchen Koeffizienten besitzt $x^6y^7$ in $(x+y)^{13}$?\\\\
        Gemäß Satz 8. (Binomialtheorem): $(x+y)^{13} = \sum_{k=0}^{13} \binom{13}{k}x^{13-k}y^k$.\\\\
        Dann $x^6y^7$ besitzt $\binom{13}{7} = 1716$.

        % Task (i)
        \item Welchen Koeffizienten besitzt $x^3y^3z^7$ in $(x+y+z)^{13}$?\\\\
        Nach Binomialtheorem:
        \begin{align*}
        (x+y+z)^{n} &= \sum_{k=0}^{n} \binom{n}{k}x^{n-k}(y+z)^k \\
        &= \sum_{k=0}^{n} \binom{n}{k}x^{n-k}\sum_{j=0}^{k}\binom{k}{j}y^{k-j}z^j \\
        &= \sum_{k=0}^{n}\sum_{j=0}^{k} \binom{n}{k}\binom{k}{j}x^{n-k}y^{k-j}z^j
        \end{align*}
        Dann nach Voraussetzung: 
        \[(x+y+z)^{13} = \sum_{k=0}^{13}\sum_{j=0}^{k} \binom{13}{k}\binom{k}{j}x^{13-k}y^{k-j}z^j\]
        Folglich $x^3y^3z^7$ besitzt $\binom{13}{10}\cdot\binom{10}{7} = 286 \cdot 120 = 34440$.

        % Task (j)
        \item Wie können Sie $\sum_{k=0}^n \binom{n}{k}^2$ vereinfachen?\\\\
        Gemäß Satz 11.: $\sum_{k=0}^n \binom{n}{k}\cdot\binom{n}{k} = \binom{2n}{k}$.\\\\
    \end{enumerate}
    
    \section*{Kreativitat:} 
    Zeigen Sie mittels vollständiger Induktion:
    \[\binom{2n}{n}\geq \frac{4^n}{2\sqrt n}\]
    \begin{align*}
        &IA: n=1: \binom{2}{1}\geq \frac{4}{2} = 2\\ % & is for vertical align of IA and IS
        &IV: \binom{2n}{n}\geq \frac{4^n}{2\sqrt n}\\
        &IS: \binom{2(n+1)}{n+1}=\binom{2n+2}{n+1}
        = \binom{2n+1}{n} + \binom{2n+1}{n+1}
        = \binom{2n}{n-1} + \binom{2n}{n} + \binom{2n}{n} + \binom{2n}{n+1} &\textrm{(1)}\\
    \end{align*}
    % pay attention about \hspace{lenght} command
    \[\hspace{30pt}\binom{2n}{n-1} + 2\cdot\binom{2n}{n} + \binom{2n}{n+1} \geq \frac{4^{n+1}}{2\sqrt {n+1}}\\
    \iff 2\cdot\Bigg(\binom{2n}{n+1} + \binom{2n}{n}\Bigg) \geq 2\cdot\frac{4^n}{\sqrt {n+1}} 
    \textrm{\hspace{40pt}(2)}\\\]
    nach $IV$: \[ \binom{2n}{n+1} + \frac{4^n}{2\sqrt{n}} \geq \frac{4^n}{\sqrt {n+1}} \]\\
    Da die linke Seite deutlich grösser als rechte ist, ist die Ungleichung damit bewiesen.\\
    Bemerkung:
    \[ \frac{2n!}{(n-1)!(2n-n+1)!} = \binom{2n}{n-1} = \binom{2n}{n+1} = \frac{2n!}{(n+1)!(2n-n-1)!}\\\]
    \section*{Transfer:}
    \begin{enumerate}[label=(\alph*)]
        % Task (a) 
        \item Wie groß ist $\binom{a^n}{a^m}$? \\\\
        Hier haben wir einfach Kombination (Satz 6. (ungeordnet, ohne Zurucklegen)) mit $|a^n| = n$ und 
        $|a^m| = m$. Wir nehmen $m$ aus $n$ a's und folglich haben: 
        \[\binom{n}{m}\]

        % Task (b) 
        \item Wie groß ist $\binom{a^{n_1}b^{n_2}}{a^{m_1}b^{m_2}}$? \\\\
        Hier haben wir eine Folge von $a$ und $b$. Da koennen wir erstens Variationen von $a$ und dann 
        Variationen von $b$ (gemäß Unterabsatz a) berechnen. Ein Produkt von diese zwei Variationen ist 
        eine Variationen von die gesamte Folge. Folglich:
        \[\binom{n_1}{m_1} \cdot \binom{n_2}{m_2} \]
    \end{enumerate} 
\end{document}