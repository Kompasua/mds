% Math symbols and examples: http://en.wikibooks.org/wiki/LaTeX/Mathematics
% Also useful link: http://en.wikibooks.org/wiki/LaTeX/Advanced_Mathematics

% This document is used for title generating of exercise solutions. 
% Also it used for generating styles and environmentof documents.
% To add this title in document add next line:
% % This document is used for title generating of exercise solutions. 
% Also it used for generating styles and environmentof documents.
% To add this title in document add next line:
% % This document is used for title generating of exercise solutions. 
% Also it used for generating styles and environmentof documents.
% To add this title in document add next line:
% \input{../template.tex}
\documentclass[a4paper]{article}
\usepackage[utf8]{inputenc}
\usepackage[top=1.5cm]{geometry}
\usepackage{amssymb}
\usepackage{enumitem}
\usepackage{amsmath}
\usepackage{fancyhdr}


\allowdisplaybreaks

\DeclareMathOperator{\ggT}{ggT}
\DeclareMathOperator{\kgV}{kgV}

\pagestyle{fancy}
\fancyhf{}
\rhead{Anton Bubnov, Yevgen Kuzmenko}
\lhead{Mathematik: Diskrete Strukturen}
\cfoot{\thepage}

\title{Mathematik: Diskrete Strukturen \\ \Large Lösungsblatt}
\author{Anton Bubnov, Yevgen Kuzmenko}

\documentclass[a4paper]{article}
\usepackage[utf8]{inputenc}
\usepackage[top=1.5cm]{geometry}
\usepackage{amssymb}
\usepackage{enumitem}
\usepackage{amsmath}
\usepackage{fancyhdr}


\allowdisplaybreaks

\DeclareMathOperator{\ggT}{ggT}
\DeclareMathOperator{\kgV}{kgV}

\pagestyle{fancy}
\fancyhf{}
\rhead{Anton Bubnov, Yevgen Kuzmenko}
\lhead{Mathematik: Diskrete Strukturen}
\cfoot{\thepage}

\title{Mathematik: Diskrete Strukturen \\ \Large Lösungsblatt}
\author{Anton Bubnov, Yevgen Kuzmenko}

\documentclass[a4paper]{article}
\usepackage[utf8]{inputenc}
\usepackage[top=1.5cm]{geometry}
\usepackage{amssymb}
\usepackage{enumitem}
\usepackage{amsmath}
\usepackage{fancyhdr}


\allowdisplaybreaks

\DeclareMathOperator{\ggT}{ggT}
\DeclareMathOperator{\kgV}{kgV}

\pagestyle{fancy}
\fancyhf{}
\rhead{Anton Bubnov, Yevgen Kuzmenko}
\lhead{Mathematik: Diskrete Strukturen}
\cfoot{\thepage}

\title{Mathematik: Diskrete Strukturen \\ \Large Lösungsblatt}
\author{Anton Bubnov, Yevgen Kuzmenko}


\begin{document}
	\maketitle
	\section*{Vertiefung:}
	\begin{enumerate}[label=(\alph*)]
		% Task (a)
		\item  Bestimmen Sie $ \bmod(5^{31} \cdot 2^{789}-23^{23}, 10) $. \\
		Bemerkung: die h\"ohere Potenzen werden in Potenzen von 2 augeteilt. F\"ur Modulus Rechnenregeln wird Theorem 1.2 (BM) benutzt. 
		\begin{align*}
			% part 1
			\bmod(5^{31},10) &= 5  \textrm{\qquad da} \bmod(5^n,10)=5, \forall n \in \mathbb{N}, n\geq1
			\\\\
			% part 2
			\bmod(2^{789},10) &= \bmod(2^{512}\cdot2^{256}\cdot2^{16}\cdot2^4\cdot2^1,10) 
			\\&= \bmod(\bmod(2^{512},10)\cdot\bmod(2^{256},10)\cdot\bmod(2^{16},10)\cdot\bmod(2^4,10)\cdot\bmod(2^1,10),10)
			\\&= \bmod(6\cdot 6\cdot 6\cdot 6\cdot 2, 10) \textrm{\qquad da} \bmod(2^4,10)=6
			\\&= \bmod(6\cdot2,10) 
			\\&= 2 \\\\
			% part 3
			\bmod(-23^{23},10) &= \bmod(23^{16}\cdot23^{4}\cdot23^{2}\cdot(-23)^1,10) 
			\\&= \bmod(9\cdot 7, 10) \textrm{\qquad da} \bmod(23^2,10)=9 \textrm{ und } \bmod(23^4,10)=1
			\\&= \bmod(63,10) 
			\\&= 3 \\\\
			% result
			\bmod(5^{31} \cdot 2^{789}-23^{23}, 10) &= \bmod(5 \cdot 2 + 3,10)	
			\\&= \bmod(13,10) 
			\\&= 3
		\end{align*}

		% Task (b)
		\item  Bestimmen Sie $ \bmod(5^{31} \cdot 2^{789}-23^{23}, 11) $. \\
		Bemerkung: Rechnenweg ist gleich wie in Punkt a.
		\begin{align*}
			% part 1
			\bmod(5^{31},11) &= \bmod(5^{30}\cdot5,11)
			\\&= 5 \\\\
			% part 2
			\bmod(2^{789},11) &= \bmod(2^{512}\cdot2^{256}\cdot2^{16}\cdot2^4\cdot2^1,11)
			\\&= \bmod(9\cdot 9\cdot 9\cdot 9\cdot 2, 11)
			\\&= \bmod(432,11)
			\\&= 6 \\\\
			% part 3
			\bmod(-23^{23},11) &= \bmod(23^{16}\cdot23^{4}\cdot23^{2}\cdot(-23)^1,11)
			\\&= \bmod(4\cdot 3\cdot 5\cdot 7, 11)
			\\&= \bmod(420,11)
			\\&= 9 \\\\ 
			% result
			\bmod(5^{31} \cdot 2^{789}-23^{23}, 11) &= \bmod(5 \cdot 6 + 9,11)
			\\&= \bmod(39,11)
			\\&= 6 
		\end{align*}

		% Task (c)
		\item  Bestimmen Sie $ \bmod(7^{31} \cdot 2^{789}, 10) $. \\
		Bemerkung: Rechnenweg ist gleich wie in Punkt a.
		\begin{align*}
			% part 1
			\bmod(7^{31},10) &= \bmod(7^{16}\cdot7^{8}\cdot7^{4}\cdot7^{2}\cdot7^{1},10)
			\\&= \bmod(1\cdot1\cdot1\cdot9\cdot7,10)
			\\&= \bmod(63,10)
			\\&= 3 \\\\
			% part 2
			\bmod(2^{789},10) &= \bmod(2^{512}\cdot2^{256}\cdot2^{16}\cdot2^4\cdot2^1,10)
			\\&= \bmod(6\cdot 6\cdot 6\cdot 6\cdot 2, 10)
			\\&= \bmod(12,10)
			\\&= 2 \\\\
			% result
			\bmod(7^{31} \cdot 2^{789}, 10) &= \bmod(3 \cdot 2,10)
			\\&= \bmod(6,10)
			\\&= 6 
		\end{align*}

		% Task (d)
		\item Bestimmen Sie kgV(178, 144).
		\begin{align*}
			& 178 = 2\cdot89 \\
			& 144 = 2^4\cdot3^2 \\
			&\kgV(178, 144) = 2^4\cdot3^2\cdot89 = 12816
			% comment
		 	&\textrm{(nach Lemma 1.5 (BM))}
		\end{align*}

		% Task (e)
		\item Bestimmen Sie ggT(12877480, 24145275).
		\begin{align*}
			\ggT(12877480, 24145275) &= \ggT(24145275-12877480, 12877480)
			\\& = \ggT(12877480-11267795,11267795)
			\\& = \ggT(11267795-1609685,1609685)
			\\& = 1609685
			% comment
		 	&\textrm{(nach Lemma 1.8 (BM))}
		\end{align*}

		% Task (f)
		\item Wie sieht der $\frac{12877480}{24145275}$ zu äquivalente teilerfremde Bruch aus?
		\begin{align*}
		\end{align*}

		% Task (g)
		\item Wie viele Funktionen $f : \{0, 1, 2, 3\}^n \to \{0, 1, 2\}$ gibt es, die genau einmal den Funktionswert 0 annehmen?
		\begin{align*}
		\end{align*}

		% Task (h)
		\item Wie viele Funktionen $f : \{0, 1, 2, 3\}^n \to \{0, 1, 2\}$ gibt es, die genau zweimal den Funktionswert 0 annehmen?
		\begin{align*}
		\end{align*}
		
	\end{enumerate} 
	
	\section*{Transfer:}
	\begin{enumerate}[label=(\alph*)]
		\item 8 Symbole
		\begin{enumerate}[label=\bfseries Schritt \arabic*:]
			% Schritt 1
			\item Wir w\"ahlen beliebig zwei S\"atze. Um L\"ange 8 zu bekommen mu\"ssen wir insgesamt 4 W\"ortern haben,
			weil andere 4 Symbolen zus\"atzliche Zeichen sind.

			% Schritt 2
			\item Da es in Alphabet 26 Buchstaben gibt, k\"onnen wir von 4 kleine Buchstaben so viel Kombinationen machen:
			\begin{align*}
				26^4=456976
			\end{align*}

			% Schritt 3
			\item Wenn jede kleine Buchstabe eventuell auch gr\"osse sein kann, k\"onnen wir von 4 kleine und grosse Buchstaben so viel Kombinationen machen:
			\begin{align*}
				(26\cdot2)^4=7311616
			\end{align*}

			% Schritt 4
			\item In Reihe von 4 Buchstaben k\"onnen wir Ziffern in 7 Varianten stellen:
			\begin{align*}
				& X \in {0,1,...,9} \\
				& \{a|bcd\} \to \{Xa|Xbcd\},\{aX|bcdX\} \\
				& \{ab|cd\} \to \{Xab|Xcd\},\{aXb|cXd\}, \{abX|cdX\} \\
				& \{abc|d\} \to \{Xabc|Xd\},\{abcX|dX\}
			\end{align*}
			Erster Zahl kann 10 Ziffer sein und zweite 9 Ziffern. Also haben wir so viele Varianten f\"ur 2 Verschiedene 
			Ziffern und 7 Satz Kombinazionen:
			\begin{align*}
				& 10\cdot9\cdot7 = 630 \\
				& 7311616\cdot630 = 4606318080
			\end{align*}

			% Schritt 5
			\item N\"achstens haben wir 3 M\"oglichkeiten 6 Symbolen zu stellen:
			\begin{align*}
				& 3\cdot6=18 \\
				& 4606318080\cdot18 = 82913725440
			\end{align*}

			% Schritt 6
			\item Dann haben wir 22 Symbole am Anfang oder am Ende:
			\begin{align*}
				& 22\cdot2=44 \\
				& 82913725440\cdot44 = 3648203919360
			\end{align*}
		\end{enumerate}
		\rule{14cm}{0.4pt} \\
		Da in diese Aufgabe \"andert sich nur der Zahl der Symbole, kann man dazu ein Formel vervenden. 
		Anzahl der Kombinationen in Schritt 4 kann man so berechnen:
			\[k = \Big(\frac{n}{2}\Big)^2+n-1 \text{, wo n ist Anzahl der Symbole in Passwort minus 4.}\]
		Dann gem\"as der Informaton aus Schritten in (a) 1-6 bauen wir solche Funktion:
			\[f(n) = 52^n\cdot10\cdot9\cdot\Big(\Big(\frac{n}{2}\Big)^2+n-1\Big)\cdot18\cdot44\]
		\rule{14cm}{0.4pt} \\
		\item 10 Symbole: f(6) = 19729486795898880
		\item 12 Symbole: f(8) = 87644017343610224640
		\item 14 Symbole: f(10) = 350332190369658678804480
		\item 16 Symbole: f(12) = 1309500512049975946232463360
	\end{enumerate}
\end{document}