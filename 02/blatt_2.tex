% Math symbols and examples: http://en.wikibooks.org/wiki/LaTeX/Mathematics
% Also useful link: http://en.wikibooks.org/wiki/LaTeX/Advanced_Mathematics

% This document is used for title generating of exercise solutions. 
% Also it used for generating styles and environmentof documents.
% To add this title in document add next line:
% % This document is used for title generating of exercise solutions. 
% Also it used for generating styles and environmentof documents.
% To add this title in document add next line:
% % This document is used for title generating of exercise solutions. 
% Also it used for generating styles and environmentof documents.
% To add this title in document add next line:
% \input{../template.tex}
\documentclass[a4paper]{article}
\usepackage[utf8]{inputenc}
\usepackage[top=1.5cm]{geometry}
\usepackage{amssymb}
\usepackage{enumitem}
\usepackage{amsmath}

\allowdisplaybreaks

\DeclareMathOperator{\ggT}{ggT}
\DeclareMathOperator{\kgV}{kgV}

\title{Mathematik: Diskrete Strukturen \\ \Large Lösungsblatt}
\author{Anton Bubnov, Eugen Kuzmenko}

\documentclass[a4paper]{article}
\usepackage[utf8]{inputenc}
\usepackage[top=1.5cm]{geometry}
\usepackage{amssymb}
\usepackage{enumitem}
\usepackage{amsmath}

\allowdisplaybreaks

\DeclareMathOperator{\ggT}{ggT}
\DeclareMathOperator{\kgV}{kgV}

\title{Mathematik: Diskrete Strukturen \\ \Large Lösungsblatt}
\author{Anton Bubnov, Eugen Kuzmenko}

\documentclass[a4paper]{article}
\usepackage[utf8]{inputenc}
\usepackage[top=1.5cm]{geometry}
\usepackage{amssymb}
\usepackage{enumitem}
\usepackage{amsmath}

\allowdisplaybreaks

\DeclareMathOperator{\ggT}{ggT}
\DeclareMathOperator{\kgV}{kgV}

\title{Mathematik: Diskrete Strukturen \\ \Large Lösungsblatt}
\author{Anton Bubnov, Eugen Kuzmenko}


\begin{document}
	\maketitle
	\section*{Vertiefung:}
	\begin{enumerate}[label=(\alph*)]
		% Task (a)
		\item  Wie viele Verlosungen von 5 identischen Kaffeemaschinen unter 25 Teilnehmern gibt es?\\\\
		Ziehen ohne Zurücklegen, ohne Reihenfolge.\\	
		$${n \choose k}= {25 \choose 5}= \frac{25!}{5!(25-5)!}= 
		\frac{20!\cdot 21 \cdot 22 \cdot 23 \cdot 24 \cdot 25}{5! \cdot 20!} = 53 \; 1230 $$
		
		% Task (b)
		\item Wie viele Möglichkeit gibt es, genau 7 Chips auf die drei Felder 1-12, 13-24, 25-36 
        beim Roulette zu legen?\\\\ 
		Wir ziehen die Felder ohne Reihenfolge mit Zur\"ucklegen.\\
		$n=3; \; k=7$ \\
		$$ \binom{n+k-1}{k}= \binom{3+7-1}{7}= \binom{9}{7}=
		\frac{9!}{7!(2!)}= \frac{8*9}{2}=36 $$

		% Task (c)
		\item Wie viele Binärzahlen der Länge 8 beginnen mit einer 0 oder enden mit 11?\\\\
		Da am Anfang haben wir 0, bleib f\"ur die Reste 7 Stellen, also $2^7 = 128$ Varianten.\\
		Da am Ende haben wir zwei 1, bleib f\"ur die Reste 6 Stellen, also $2^6 = 64$ Varianten.\\
        Nach Satz 6. (geordnet, mit Zur\"ucklegen)

		% Task (d)
		\item Sie haben 5 Informatik-Bücher, 4 Mathematik-Bücher und 3 Philosophie-Bücher zur
		Auswahl. Wie viele Möglichkeiten gibt es, auf eine Reise zwei Bücher aus verschiedenen
        Themenbereichen mitzunehmen?\\\\
        Nach Satz 6. (ungeordnet, ohne Zur\"ucklegen) es gibt 3 Kombinationen von F\"achern und zwar
        Inf. und Mathe, Inf. und Phil., Mathe und Phil.\\
        Gem\"ass jeden Fall haben wir insgesamt so viel Kombinationen:
        \[5 \cdot 4 + 5 \cdot 3 + 4 \cdot 3 = 47\]

        % Task (e)
        \item In einer Gruppe von 11 Personen sind 5 Vorstandsposten (ohne Personalunion) für jeweils
        eine Person zu vergeben: Präsidentin, Vizepräsidentin, Geschäftsführerin, stellvertreten-
        de Geschäftsführerin, Schatzmeisterin. Wie viele verschiedene Vorstände sind möglich?\\\\

        % Task (f)
        \item Wie viele verschiedene Wörter können Sie aus dem Wort PICHICHI bilden?\\\\
        In dieser Wort haben wir 8 Buchstaben. Da aber manche Buchstaben sich viederholen: 
        I=3, C=2, H=2 - Mal. Da wir die Umstellung der gleiche Buchstaben nicht unterscheiden, haben wir:
        \[\frac{n!}{k_1!\cdot k_2!\cdot\ldots\cdot k_n!} = \frac{8!}{3!\cdot2!\cdot2!} = 1680\]
		
	\end{enumerate}
\end{document}