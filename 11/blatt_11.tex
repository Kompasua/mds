% Math symbols and examples: http://en.wikibooks.org/wiki/LaTeX/Mathematics
% Also useful link: http://en.wikibooks.org/wiki/LaTeX/Advanced_Mathematics

% This document is used for title generating of exercise solutions. 
% Also it used for generating styles and environmentof documents.
% To add this title in document add next line:
% % This document is used for title generating of exercise solutions. 
% Also it used for generating styles and environmentof documents.
% To add this title in document add next line:
% % This document is used for title generating of exercise solutions. 
% Also it used for generating styles and environmentof documents.
% To add this title in document add next line:
% \input{../template.tex}
\documentclass[a4paper]{article}
\usepackage[utf8]{inputenc}
\usepackage[top=1.5cm]{geometry}
\usepackage{amssymb}
\usepackage{enumitem}
\usepackage{amsmath}

\allowdisplaybreaks

\DeclareMathOperator{\ggT}{ggT}
\DeclareMathOperator{\kgV}{kgV}

\title{Mathematik: Diskrete Strukturen \\ \Large Lösungsblatt}
\author{Anton Bubnov, Eugen Kuzmenko}

\documentclass[a4paper]{article}
\usepackage[utf8]{inputenc}
\usepackage[top=1.5cm]{geometry}
\usepackage{amssymb}
\usepackage{enumitem}
\usepackage{amsmath}

\allowdisplaybreaks

\DeclareMathOperator{\ggT}{ggT}
\DeclareMathOperator{\kgV}{kgV}

\title{Mathematik: Diskrete Strukturen \\ \Large Lösungsblatt}
\author{Anton Bubnov, Eugen Kuzmenko}

\documentclass[a4paper]{article}
\usepackage[utf8]{inputenc}
\usepackage[top=1.5cm]{geometry}
\usepackage{amssymb}
\usepackage{enumitem}
\usepackage{amsmath}

\allowdisplaybreaks

\DeclareMathOperator{\ggT}{ggT}
\DeclareMathOperator{\kgV}{kgV}

\title{Mathematik: Diskrete Strukturen \\ \Large Lösungsblatt}
\author{Anton Bubnov, Eugen Kuzmenko}

\DeclareMathOperator{\kgV}{kgV}

\begin{document}
    \maketitle
    \section*{Vertiefung:}
    \begin{enumerate}[label=(\alph*)]
        % Task (a)
        \item Ist $\langle \mathbb{R}_{>0} , f \rangle$ mit der Verknüpfung $f : (x,y) \mapsto  \log_y x$ eine Algebra? \\
        Ja, denn $\mathbb{R}_{>0}$ stellt eine nichtleere Menge dar. Der Operator $f$ ist zudem eine Abbildung, 
        der Logarithmus, der für alle $\mathbb{R}_{>0}$ gültig ist. Definition 4.1 ist somit erfüllt.
        
        % Task (b)
        \item Ist $\langle \{n| n \in \mathbb{N} \textrm{ ist eine ungerade Zahl} \}, f \rangle$ mit dem dreistelligen 
        Operator $f : (x,y,z) \mapsto \bmod(x^2 + 2y, z + 2)$ eine Algebra? \\
        Nein. Betrachte hierzu den Fall $x = 1, y = 1, z = 1$. In diesem Fall würde gelten 
        $$\bmod(1^2 + 2 \cdot 1, 1 + 2 ) = \bmod(3,3) = 0$$ Die Null ist aber eine gerade Zahl und muss auch 
        in $n$ liegen. Aus diesem Widerspruch folgt, dass hier keine Algebra vorliegt.
        
        % Task (c)
        \item
         Ist$\langle \mathbb{Z}_8 \setminus \{0\}, \cdot_k \rangle $ eine Algebra – und wenn nein, warum nicht? \\
         Nach dem Beispiel auf Seite 68 ist die Multiplikation $\cdot_k$ auf der Menge $\mathbb{Z}_k := \{0, 1, ..., k-1\}$ definiert. Folglich kann hier keine Algebra definiert sein.
        % Task (d)
        \item
        Bestimmen Sie eine Teilmenge $A \subseteq \mathbb{Z}_{10} \setminus \{0\} $ mit maximaler Anzahl von Elementen, sodass$\langle A, \cdot_{10} \rangle$ eine Algebra ist. 
        $\langle \mathbb{Z} \setminus \{10, 5\}, \cdot_{10} \rangle $ ist eine Algebra mit maximaler Anzahl von Elementen, da kein  Produkt mehr durch 10 oder 5 teilbar ist.
        
        % Task (e)
        \item 
        Das neutrale Element ist 1.
        Da Algebra $\langle\mathbb{N}_+; kgV \rangle$ kommutativ ist, reicht die Linksneutralität von 1. Sei $x \in \mathbb{ N}_+, n \setminus \{1\}$ beliebig aber fest.  Da 1 kein Vielfaches von x ist, gilt also $kgV (1, x) = 1 \cdot x =  x$. Da außerdem gilt $kgV (1, 1) = 1$ folgt, dass das neutrale Element 1 ist.
        % Task (f)
        \item Wie viele inverse Elemente besitzt 2 in der Algebra $\langle \mathbb{N}_+ , \kgV \rangle$?\\
        Überhaupt kein einziges.
        
        % Task (g)
        \item 
        
        % Task (h)
        \item Können Sie die gegebene Verknüpfungstabelle für $\circ$ so ergänzen, dass $\circ$ assoziativ 
        auf \{a, b, c, d\} ist?\\
        Das geht nicht auf.
        
        % Task (i)
        \item
        
        % Task (j)
        \item         
        
    \end{enumerate}
    \section*{Kreativität:}
    \begin{enumerate}[label=(\alph*)]
    	%Task (a)
    	\item Task a
    \end{enumerate}
    \section*{Transfer:}
    \begin{enumerate}[label=(\alph*)]
    	\item
        \begin{align*}
            (R \Join S) \Join T &= \{(a,d) \, | \, \exists\, c \in C : (a,c) \in R \Join S \land (c,d) \in T \}\\
            &=\{(a,d) \, | \, \exists\, c \in C : (a,c) \in \{(a,c) \,|\, \exists\, b \in B : (a,b) \in R \land (b,c) 
            \in S \}  \land (c,d) \in T \}\\
            &=\{(a,d) \, | \, \exists\, c \in C \, \exists\, b \in B : (a,b) \in R \land (b,c) \in S   \land (c,d) \in T \}\\
            &=\{(a,d) \, |  \, \exists\, b \in B \, \exists\, c \in C :  (b,c) \in S   \land (c,d) \in T \land (a,b) 
            \in R  \}\\
            &=\{(a,d) \, |  \, \exists\, b \in B : (b,d) \in \{(b,d) \,|\, \exists\, c \in C :  (b,c) \in S   
            \land (c,d) \in T\} \land (a,b) \in R  \}\\
            &=\{(a,d) \, |  \, \exists\, b \in B : (b,d) \in S\Join T \land (a,b) \in R  \}\\
            &=\{(a,d) \, |  \, \exists\, b \in B : (a,b) \in R \land (b,d) \in S\Join T  \}\\
            &=R \Join (S \Join T)
        \end{align*}
        \item
        Die Kardinalitet von alle Joins in die Tabelle ist 8. Wenn wir folglich so klammern:
        $$  (E \Join E) \Join (E \Join E) \Join W$$
        dann muss die $(E \Join E)$ nur einmal usgefuert werden. Dann die Gesamtkardinalitaet ist 
        $$3\cdot 8 = 24$$
    \end{enumerate}
\end{document}






